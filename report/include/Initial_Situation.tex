\section{Initial Situation}\label{section:initial_situation}

\subsection{Formalization Of The Problem}
\subsubsection{The edge-length ratio}
Let $\Gamma_G$ be a given planar polyline drawing. The length of an edge is defined as the sum of $k+1$ line segments, induced by $k$ bends. $l_{\max}$ is the length of the longest edge in $\Gamma_G$, $l_{\min}$ is the minimal Euclidian distance between two adjacent vertices in $\Gamma_G$. Then, the edge-length ratio $r$ of $\Gamma_G$ is defined as:
\begin{align}
	r_{\Gamma_G} = \frac{l_{\max}}{l_{\min}} 
\end{align}
It trivially holds, that $r\geq1$, since the length of every polyline with at least one bend between two vertices is naturally longer than the Euclidian distance between those. $r$ is said to be \emph{optimal }if $r=1$. Then, all the edges in a drawing are straight-lines and of the same length.
\subsubsection{Upper bound of the ratio}
There exist multiple straight-line drawing algorithms which produce a drawing for a planar graph in area $\mathcal{O}(n)\times\mathcal{O}(n)$.
% TODO cite some algorithms
The area consumption of a straight-line drawing directly induces the bounds for the ratio. Let $k\times k$ be the area consumption of a bounding square $\Gamma_G$ is drawn on, $k\in \mathcal{O}(n)$. The maximal length of a straight-line is then bound by $\sqrt{2}k$, from one corner of the grid to the diagonal opposing one, while $l_{\min}$ might value 1 \UL. The ratio therefore values $\sqrt{2}k \in \mathcal{O}(n)$ in the worst case.\\
This automatically gives an upper bound for any poly-line drawing $\Omega_G$ since a straight-line drawing can be seen as a polyline drawing with zero bends. Including bends in a straight-line drawing enables the possibility to reposition vertices in order to maximize the Euclidian distances.

% TODO NP-Hardness

\subsubsection{\NP-hardness}

% 2-trees state of the art

\subsection{On the edge-length ratio of 2-trees}

The class of $2$-trees is of particular interest in this thesis due to their balance in restricted properties on the one hand, and having non-trivial approaches and results for general problems on the other hand. This effect is pointed out by previous results regarding the edge-length ratio.\\
$2$-trees are biconnected, but not triconnected. This property implies a high amount of possible embeddings for a given $2$-tree $G$, since parallel subgraphs can be permuted and flipped. Therefore, finding drawings with an optimization regarding a specific problem require combinatorial and algorithmic approches.\\
It was proven that the ratio of straight-line drawings of 2-trees is unbounded, meaning, that for a given constant $r$, there exists a sufficiently large 2-tree $G$ with $r_G > r$ over all its straight-line drawings. One result states that the ratio lies in the gap between the lower bound $\Omega(\log n)$ and the upper bound $\mathcal{O}(n^{0.695})$ \cite[P. 2]{edge-length-ratio-2tree}.

% symposium challenge

\subsection{The Symposium Challenge}

% TODO intro neu schreiben

%There has been recent attention to the edge-length ratio of a planar drawing, which describes the ratio between the \emph{length of the longest edge} and the \emph{Euclidian distance between two adjacent vertices} in a drawing. 
%\newline This year, the main topic addresses an optimization problem, namely the minimization of the edge-length ratio of polyline drawings of planar, undirected graphs on a fixed grid. For a polyline edge, the edge-length is the sum of the line segment lengths.
\bigbreak The input consists of a JSON file with the following entries:
\begin{description}
	\item[nodes] Every node has an unique ID value between 0 and the amount of nodes - 1, a value for the $x$ and $y$ coordinate each, delimited by the width and height
	\item[edges] Every edge has an ID for source and destination each and an optional list of bend points, specified in $x$ and $y$ coordinate
	\item[width (optional)] The maximum $x$-coordinate of the grid. If unspecified, the width is set to 1,000,000.
	\item[height (optional)] The maximum $y$ coordinate of the grid. If unspecified, the height is set to 1,000,000.
	\item[bends] The maximum number of bends allowed per edge
\end{description}
The results of the optimization are also JSON files. The planarity of the graph shall be preserved and the ratio minimized by relocation of the nodes.\\
For teams participating with their own tools, an embedding might not be given with the input. For participants working manually, an embedding is already given beforehand.

% TODO neues beispiel

% \subsubsection{A small example - a triangle}

%This example will illustrate the issue of finding a drawing with an optimal edge-length ratio. Consider $K_3$, the complete graph with three vertices. $K_3$ has got one outerface and one inner face, in shape of a triangle. In order to find a drawing of $K_3$ with an optimal ratio, we will see that one bend per edge is mandatory.\\
%A triangle with straight-lines as edges with length $l$ will have a height of $\frac{\sqrt{3}}{2}l$ in order to be optimal regarding the ratio. Unfortunately, $\sqrt{3}$ is not a rational number. There do not exist two integers in order to express $\sqrt{3}$ as a fraction. As a consequence, there does not exist any combination of coordinates on a grid in order to draw the $K_3$ without any bends and an optimal ratio.
%\begin{figure}[H]
%	\centering
%	\begin{subfigure}{0.6\linewidth}
%		\centering
%		\includegraphics[width=0.7\textwidth,page=1]{drawings/small_example.pdf}
%	\end{subfigure}
%	\caption{There is no grid point for an optimal drawing without any bends}
%\end{figure}
%But, when introducing one bend per edge, there exists a drawing with an optimal ratio. 
%\begin{figure}[H]
%	\centering
%	\begin{subfigure}{0.6\linewidth}
%		\centering
%		\includegraphics[width=0.7\textwidth,page=2]{drawings/small_example.pdf}
%	\end{subfigure}
%	\caption{With one bend, $K_3$ admits an optimal drawing}
%\end{figure}
%Allowing bends will help minimizing the ratio of a drawing. 