\section{Maximal Series-Parallel Graphs}\label{section:SP-graphs}

Creating a compact layered polyline drawing for maximal outerplanar graphs will prove to be useful when drawing maximal SP-graphs. Every maximal SP-graph contains a maximal outerplanar subgraph. 

\subsection{Properties Of Maximal Series Parallel Graphs}

% 2-tree stuff

\begin{lemma}
	For every maximal SP-graph $G$, there exists a maximal outerplanar subgraph $G'$ that is larger than every other maximal outerplanar subgraph.
\end{lemma}
\begin{proof}
	Consider the tree decomposition $(T,W)$ of $G$. Add the root of $T$ to $T'$ and its bag to $W'$. From the root of $T$, pick the three children which root the subtrees with the maximum height and size and their respective bags. Then, recursively pick the two children which root the subtrees with the maximum height and size and add them to $T'$ with their respective bags to $W'$. This procedure terminates some leaves of $T$.\\
	Every vertex of the resulting $T'$ is of degree maximum 3 and by Lemma \ref{l:outerplanar_tree_decomposition}, the graph $G'$ induced by $T$ is maximal outerplanar. Since during this procedure, the nodes with the maximum subtrees are chosen and the properties described in Lemma \ref{l:outerplanar_TD_properties} hold, there exists no other maximal outerplanar subgraph of $G$ which is larger. 
\end{proof}

\begin{lemma}
	Let $G$ be a maximal SP-graph with tree decomposition $(T,W)$ and $G'$ any maximal SP subgraph of $G$ with tree decomposition $(T',W')$. Then, it holds:
	\begin{enumerate}
		\item $(T',W') \subseteq (T,W)$
		\item There exists a vertex in $T\setminus T'$ whose bag shares exactly two vertices of $G$ with the bag of the root of $T'$
	\end{enumerate}
\end{lemma}
\begin{proof}
	\begin{itemize}
		\item Since the tree decomposition of a maximal SP-graph is unique, any maximal SP subgraph $G'$ of $G$ is represented in $T$ by $T'$.
		\item Let $t$ be the root of $T'$. Consider the parent of $t$ in $T$. This property holds between those bags by Lemma \ref{l:outerplanar_two_vertices_subtree_TD_2}.
	\end{itemize}
\end{proof}

% drawing stuff

\subsection{Drawing Algorithm For Maximal Series Parallel Graphs With Two Bends}

A maximal SP-graph $G$ with its tree decomposition $(T,W)$ will be evaluated for its largest maximal outerplanar subgraph $G'$ with tree decomposition $(T',W')$. $\tilde{T}$ describes the difference of $T$ and $T'$ and is a set of subtrees of $T$.\\
In order to create a polyline drawing similarily as described in section \ref{s:maximal_outerplanar}, it is crucial to find a suiting $SPQR$ tree derivation of a maximal SP-graphs for the box drawing algorithm \ref{al:draw_SPQR} to function. Fortunately, the function deriving a $SPQR$ tree of a tree decomposition of a maximal outerplanar graph $G'$ is also applicable for maximal SP-graphs.
\begin{lemma}
	The function deriving a $SPQR$ tree from a tree decomposition defined in Lemma \ref{l:tree_decomp_to_SPQR} is applicable to any maximal SP-graph $G$.
\end{lemma}
\begin{proof}
	Let $G$ be a maximal SP-graph with its tree decomposition $(T,W)$. Contrary to a tree decomposition of a maximal outerplanar graph, any node $T$ can have arbitrary many children. Inserting a new vertex described with a vertex in $T$ works analogously to the prior scheme. For a maximal SP-graph, the resulting $SPQR$ tree can inherit arbitrary many serial nodes attached to a parallel node.
\end{proof}

After $(T',W')$ is drawn into a box drawing $B'$, the remaining subtrees in $\tilde{T}$ will be inserted accordingly, resulting in a box drawing $B$ for $G$. Finally, a polyline drawing will be derived from $B$.

\begin{lemma}
	Let $G$ be a maximal SP-graph and $G'$ its largest maximal outerplanar graph with its box drawing $\mathcal{B}_{G'}$ created by Algorithm \ref{al:maximal_outerplanar_box_two_bends}. For every tree $t$ in $\tilde{t}$ there exists a column in $\mathcal{B}_{G'}$ where $t$ can be inserted into without destroying planarity. 
\end{lemma}
\begin{proof}
	Consider the two shared vertices $u,l \in G$ of the root of $t$ with its parent. Since the parent of $t$ is part of $G'$, $u$ and $l$ are placed on their respective layers in $\mathcal{B}_{G'}$ and connected by an edge at column $x$ with its content drawn from $G'$. Insert $t$ by slicing the box drawing so that there are sufficient free columns beginning at $x+1$ and draw $t$. This way, planarity is preserved. 
	
	\begin{figure}[H]
	\centering
	\begin{subfigure}{\textwidth}
		\centering
		\includegraphics[page=2,width=0.6\linewidth]{graphics/t_insertion_into_maximal_outerplanar.pdf}
		\caption{$t$ inserted inbetween upper and lower layer $u$ and $l$ (colored in orange)}
	\end{subfigure}
	
\end{figure}
	
\end{proof}

\begin{algorithm}[H]
	\KwIn{Maximal Series Parallel Graph $G$}
	\KwOut{Polyline Drawing $\Gamma_G$ with two bends per edge}
	\caption{\texttt{DrawMaximalSPGraph}$(G)$}\label{al:DrawMaximalSPGraph}
	$(T,W) \gets$ tree decomposition of $G$\\
	$G' \gets$ largest maximal outerplanar subgraph of $G$\\
	$(T',W') \gets$ tree decomposition of $G'$\\
	$\tilde{T} \gets T\setminus T'$\\
	$\Gamma_{G'} \gets \texttt{DrawMaximalOuterplanar}(G')$\\
	\For{$t\in \tilde{T}$}{
		 $p \gets \texttt{parent}(t\texttt{.root})$\\
		 $u \gets \Gamma_{G'}\texttt{.upperLayer}(p)$\\
		 $l \gets \Gamma_{G'}\texttt{.lowerLayer}(p)$\\
		 $\Gamma_{G'}\texttt{.Insert}(t, u, l)$
	}
	\Return $\Gamma_{G'}$
\end{algorithm}

\subsubsection{Analysis}

\begin{lemma}
	When the remaining subtrees of $\tilde{t}$ are inserted into a box drawing $\mathcal{B}_{G'}$, the area bounds do not increase asymptotically.\label{l:SP_not_increasing_area}
\end{lemma}
\begin{proof}
	Let $G$ be a maximal SP-graph with $\mathcal{B}_{G'}$ a box drawing from the largest maximal outerplanar graph of $G$ created by Algorithm \ref{al:maximal_outerplanar_box_two_bends}. For the insertion of $t\in \tilde{t}$ into $\mathcal{B}_{G'}$, consider the subtrees adjacent to the parent $p$ of $t$ in $T$. Since $t$ is not in $T'$, there exists subtree $t'$ of $T'$ adjacent to $p$ which was already drawn.\\
	If the insertion of a tree $t$ would create more layers than drawing $t'$ in the first place resulting in a new upper bound for the height of the drawing, then there would be a maximal outerplanar subgraph of $G$ containing $t'$ which is contradicting the assumption. In the width of the drawing, there are still as many columns as there are edges in $G$. The area bounds do not increase asymptotically.
\end{proof}

% Results

\begin{theorem}
	Any maximal SP-graph admits a polyline drawing in $\mathcal{O}(n^2 \log^2 n)$ area with two bends per edge and a ratio bound by $\mathcal{O}(\log^2 n)$, when the minimal distance between two layers is set to $n$.
\end{theorem}
\begin{proof}
	Let $\mathcal{B}_{G'}$ be the box drawing of the largest maximal outerplanar subgraph $G'$ of a maximal SP-graph $G$. The width of $\mathcal{B}_{G'}$ lies in $\mathcal{O}(n')$. Inserting the remaining subtrees of $\tilde{t}$ increases the width asymptotically to $\mathcal{O}(n)$ since there is at most one column per edge of $G$. Lemma \ref{l:SP_not_increasing_area} states that the height bound is asymptotically not altered and analogously to the proof of theorem \ref{th:maximal_outerplanar_log2_n_ratio}, the ratio of the resulting polyline drawing $\Gamma_G$ is bound by the amount of total layers, when the minimal distance is set to $n$.
\end{proof}
\begin{theorem}
	Any maximal SP-graph admits a polyline drawing in $\mathcal{O}(n \log^2 n)$ area with two bends per edge.
\end{theorem}
\begin{proof}
	When the minimal distance between two layers is set to a constant, the resulting height is bound by the amount of simultaneously active vertices. The resulting drawing 
\end{proof}

\subsection{Example drawing}

\begin{figure}[H]
	\centering
	\begin{subfigure}{\textwidth}
		\centering
		\includegraphics[page=10,width=0.8\linewidth]{graphics/maximal_outerplanar_example_drawings.pdf}
			\caption{Maximal SP-graph $G$ with its largest maximal outerplanar subgraph from example \ref{im:maximal_outerplanar_example_straight-line}. The differential subtrees are colored in orange}
	\end{subfigure}
	\begin{subfigure}{\textwidth}
	\centering
	\includegraphics[page=11,width=0.5\linewidth]{graphics/maximal_outerplanar_example_drawings.pdf}
		\caption{$T$ of the tree decomposition of $G$}
\end{subfigure}

\end{figure}


\begin{figure}[H]
	\centering
	\begin{subfigure}{\textwidth}
		\centering
		\includegraphics[page=12,width=0.8\linewidth]{graphics/maximal_outerplanar_example_drawings.pdf}
	\end{subfigure}
	\caption{The first differential subtree (encircled in magenta) is inserted between vertices 1 and 5 right next to the edge of $(1,5)$ (colored in cyan). Since 1 is not finished during the insertion, creating a new layer is mandatory}
\end{figure}


\begin{figure}[H]
	\centering
	\begin{subfigure}{\textwidth}
		\centering
		\includegraphics[page=13,width=0.9\linewidth]{graphics/maximal_outerplanar_example_drawings.pdf}
	\end{subfigure}
	\caption{The second differential subtree (encircled in magenta) is inserted between vertices 13 and 3 right next to the edge of $(13,3)$ (colored in cyan)}
\end{figure}