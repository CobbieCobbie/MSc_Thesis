\section{Series-Parallel Graphs}\label{section:SP-graphs}

This thesis addresses two main approaches of ratio minimization for drawings of series parallel graphs. Maximal series parallel graphs refers to the class of 2-trees % TODO cite 2-tree<>max SP graph
and are a suitable class of interest for fundamental research since any 2-tree is biconnected, but not triconnected and inherits a constant treewidth. Furthermore, any maximal outerplanar graph is a 2-tree, therefore the class of maximal outerplanar graphs is a strict subclass of the 2-trees.\\
At first, the approaches will address the maximal outerplanar graphs. After the analysis regarding the properties of the resulting drawings it will be discussed whether the approach will be suitable to be extended for 2-trees. 

\subsection{Drawing Algorithm For Maximal Outerplanar Graphs With One Bend}

At first, the attention is drawn to \emph{outerplanar graphs} as they are a subclass of SP graphs. There will be two polyline drawing algorithms presented for this class of SP graphs. The first drawing algorithm will take advantage of the already existing drawing algorithm \ref{al:k-ary_trees} for $k$-ary trees, since the weak dual graph of a maximal outerplanar graph inheits a tree structure. The second drawing algorithm for outerplanar graphs will be suited to be extended for $2$-trees, since every maximal SP graph contains a maximal outerplanar graph as a subgraph.

\subsubsection{Properties Of Outerplanar Graphs}
\begin{lemma}
	A maximal outerplanar graph $G$ inherits triangles as inner faces, except for the outerface.
\end{lemma}
\begin{lemma}\label{l:outerplanar-dual-tree-degree-3}
	The weak dual graph $G^*$ of a maximal outerplanar graph $G$ is a simple tree with maximum degree 3 for any vertex.
\end{lemma}
\begin{proof}
	The weak dual graph $G^*$ is connected since $G$ is maximal outerplanar. Suppose, that $G^*$ contains a cycle $\mathcal{C}$. Then, there exists a vertex in $G$ which is enclosed from the outerface by faces according to $\mathcal{C}$ in $G^*$ and $G$ is not outerplanar. This implies that $G^*$ must be acyclic and considering the connectedness, $G^*$ is a tree. Since any face $f$ is a triangle, the degree of $v_f$ in $G^*$ values at most three. The simplicity is derived from the maximal outerplanarity property. If there were multiple edges between vertex $v_f$ and $v_{f'}$ in $G^*$, then there would be at least one vertex in $G$ which does not lie on the outerface.
\end{proof}
\begin{lemma}
	Let $G$ be a maximal outerplanar graph with $n$ vertices and $G^*$ the dual graph excluding the outerface, a rooted tree with degree up to three for every vertex $v_f$. Then, the height of $G^*$ ranges between $\Omega(\log n)$ and $\mathcal{O}(n)$.
\end{lemma}
\begin{proof}
	Since $G$ is a planar graph, it contains $\mathcal{O}(n)$ faces. The rooted tree $G^*$ inherits the following property:
	\begin{enumerate}
		\item The root has at most three children
		\item The subtrees rooted at the children of the root are binary
	\end{enumerate}
	Placing $\mathcal{O}(n)$ vertices in three binary trees connected to a root vertex results in a height of at least $\Omega(\log n)$ due to the $k$-ary tree height property from Lemma \ref{l:k-ary-tree_log_height}. In the worst case, $G^*$ will be a chain of vertices, therefore a rooted tree with height $\mathcal{O}(n)$.
\end{proof}

\begin{observation}
	% TODO linear height with subtrees of constant height
\end{observation}

\begin{lemma}
	A maximal outerplanar graph $G$ can be extended to a maximal outerplanar supergraph $G'$.\label{l:outerplanar-supergraph}
\end{lemma}
\begin{proof}
	The vertex insertion works analougously to the recursive definition of a 2-tree. A new vertex can be added to $G$ by adding a new vertex $v_f$ in the dual graph $G^*$ so that the degree of $G^*$ is still at most 3. The newly created face $f$ must lie on the outerface and must be a triangle. Otherwise, the outerplanarity property is destroyed. 
\end{proof}

% One Bend, maximal complete outerplanar graph
When $G^*$ inherits a height of $\mathcal{O}(\log n)$, a new problem emerges. When starting drawing the root of $G^*$, new vertices are added in all directions, enclosing more and more area along the iterative drawing. 
\begin{observation}
	
\end{observation}
Figure bla illustrates this problem. 
% TODO Figure of area problem
This results in short euclidian distances relative to the longest edge, increasing the ratio.\\
\begin{definition}\label{def:complete_maximal_outerplanar}
	A maximal outerplanar graph is called \emph{complete} if its weak dual graph $G^*$ fulfills these properties:
\begin{enumerate}
	\item The root vertex has exactly three children
	\item Every other inner node has exactly two children. In other words, the subtrees adjacent to the root vertex are complete binary trees of height $h-1$
\end{enumerate}
\end{definition}

\subsubsection{Drawing A Complete Outerplanar Graph With One Bend}

A given maximal outerplanar graph can be drawn by using a drawing algorithm for its weak dual graph. In section \ref{s:k-ary_trees}, the $k$-ary tree drawing algorithm produces a straight-line drawing in $\mathcal{O}(n^2 \log n)$ area with a ratio of $1+\varepsilon,\varepsilon>0$. The drawing algorithm \ref{al:k-ary_trees} can be used with a minor modification to draw the weak dual graph of any complete maximal outerplanar graph $G$, since $G^*$ is a subtree of a $3$-ary tree with the same height.\\

\begin{algorithm}[H]
	\KwIn{A complete maximal outerplanar graph $G$}
	\KwOut{Straight-line drawing $\Gamma_{G^*}$ with nearly optimal ratio}
	\caption{\texttt{DrawOuterWeakDual($G$)}}\label{al:drawouterweakdual}
	$G^* \gets$ weak dual graph of $G$ with minimal height\\
	$h \gets$ \texttt{height($G^*$)}\\
	\texttt{root} $\gets$ $G^*$.\texttt{root}\\
	\texttt{Draw(root)}\\
	\texttt{Draw\_$3$-ary\_Children(\texttt{root},$3^h$,1)}\\
	\For{\texttt{$v \in$ root.children}}{
		\texttt{Draw\_$2$-ary\_Children($v$,$2^{h-1}$,$h-1$)}
	}
	\Return $\Gamma$
\end{algorithm}

Algorithm \ref{al:drawouterweakdual} produces a nearly optimal straight-line drawing for the weak dual graph of a complete outerplanar graph $G$. The resulting drawing $\Gamma_{G^*}$ provides assistance to draw the complete outerplanar graph $G$. Every vertex of $\Gamma_{G^*}$ serves as an anchor point for the drawing of its corresponding face in $G$.\\
Starting at the root of $G^*$, a triangle is drawn around the root in a way, that each edge of $G^*$ from the root to its three children crosses exactly one edge of the corresponding triangle face in $G$. The vertices for the triangle are placed as follows. The first vertex lies above the already drawn root with an euclidian distance to the root sufficiently high in order to preserve planarity for the remaining drawing. The other two vertices are placed inbetween the two triangles defined by the root, its $\{\text{left, right}\}$ and middle children.\\
In order to guarantee a valid vertex and bend placement at every inner node $v^*$, the drawing is stretched horizontally by a factor of three.\\
Then, the drawing algorithm iterates over the height of $G^*$. For every vertex $v^*$ of height $i\in \{1,...,h\}$ in $G^*$, two bend points are placed 1 \UL left and right from $\Gamma_{G^*}(v^*)$, and the new vertex $v$ is placed 1 \UL below from $\Gamma_{G^*}(v^*)$.\\
$v^*$ is adjacent to an edge $e^*$ defined in $G^*$ that is crossing exactly one edge $e = (v_1,v_2)$ of $G$. This crossing corresponds to the vertices defining the new face, $\{v, v_1, v_2\}$. Since $G^*$ consists of three complete binary subtrees connected to the root, using the bends to draw the new face will preserve planarity since the coordinate of every inner node inherits its unique $x$ value by construction of algorithm \ref{al:k-ary_trees}. 
% TODO illustration of the approach
This drawing approach is summarized in the following algorithm.\\
\begin{algorithm}[H]
	\KwIn{Complete outerplanar graph $G$}
	\KwOut{Polyline drawing $\Gamma_G$ with one bend per edge and ratio $\mathcal{O}(\log n)$}
	\caption{\texttt{DrawCompleteOuterplanar($G$)}}\label{al:complete_outerplanar}
	$\Gamma \gets $ \texttt{DrawOuterWeakDual($G$)}\\
	$h \gets$ \texttt{height($G^*$)}\\
	\For{$v^* \in G^*$}{
		$x(v^*) \gets 3\cdot x(v^*)$
	}
	\tcc{Draw the triangle face around the root of ${G^*}$}
	$root \gets$ $\Gamma(G^*.\texttt{root})$\\
	$l \gets$ $\Gamma(G^*.\texttt{root.leftChild})$\\
	$m \gets$ $\Gamma(G^*.\texttt{root.middleChild})$\\
	$r \gets$ $\Gamma(G^*.\texttt{root.rightChild})$\\
	$v_1 \gets$ $\Gamma$.\texttt{PlaceVertex($x(root),y(root)+h\cdot 3^h$)}\\
	$v_2 \gets$ $\Gamma$.\texttt{PlaceVertex($\frac{x(root)+x(l)+x(m)}{3},\frac{y(root)+y(l)+y(m)}{3}$)}\\
	$v_3 \gets$ $\Gamma$.\texttt{PlaceVertex($\frac{x(root)+x(m)+x(r)}{3},\frac{y(root)+y(m)+y(r )}{3}$)}\\
	$\Gamma$.\texttt{DrawLineSegment($v_1,v_2$)}\\
	$\Gamma$.\texttt{DrawLineSegment($v_1,v_3$)}\\
	$\Gamma$.\texttt{DrawLineSegment($v_3,v_2$)}\\
	
	\tcc{Iterate over the height to draw the remaining vertices of $G$}
	\For{$i \in [1..h]$}{
		\For{$v^* \in G^*$ with height $i$}{
			$b_l \gets$ $\Gamma$.\texttt{PlaceBendPoint($x(v^*) - 1,y(v^*)$)}\\
			$b_r \gets$ $\Gamma$.\texttt{PlaceBendPoint($x(v^*) + 1,y(v^*)$)}\\
			$v \gets$ $\Gamma$.\texttt{PlaceVertex($x(v^*),y(v^*)-1$)}\\
			$e^* \gets (v^*,\texttt{parent}(v^*))$\\
			$e \gets (v_1, v_2)$ edge intersecting $e^*$
			\tcp{w.l.o.g. $v_1$ ordered left of $v_2$}
			$\Gamma.$\texttt{DrawLineSegment($v,b_l$)}\\
			$\Gamma.$\texttt{DrawLineSegment($b_l,v_1$)}\\
			$\Gamma.$\texttt{DrawLineSegment($v,b_r$)}\\
			$\Gamma.$\texttt{DrawLineSegment($b_r,v_2$)}
		}
	}
	$\Gamma$\texttt{.delete($G^*$)}\\
	\Return $\Gamma$
\end{algorithm}


\subsubsection{Analysis Of Algorithm \ref{al:complete_outerplanar}}

\begin{lemma}
	Every complete outerplanar graph $G$ admits a polyline drawing $\Gamma_G$ on $\mathcal{O}(n^2 \log n)$ area with one bend per edge. The drawing is constructed in linear time and the ratio lies in $\mathcal{O}(h)$, whereas $h$ describes the height of the weak dual graph $G^*$.
\end{lemma}

\begin{proof}
The triangle around the root of $G^*$ consists of a vertex $v_1$ placed atop of the root vertex with distance $h\cdot r^h$ and two vertices $v_2,v_3$ placed at the centroids of the triangles defined by $\{\texttt{root,root.leftChild,root.middleChild}\}$ and $\{\texttt{root,root.middleChild,root.rightChild}\}$. By construction it holds, that $v_2$ and $v_3$ partition three binary subtrees rooted at the children of the root of $G^*$ by their $x$ coordinate. The placement of the vertices $v_1,v_2$ and $v_3$ guarantee exactly one intersection between an edge of $G$ and an edge of $G^*$ in $\Gamma$.\\
After stretching the drawing horizontally by a factor of three, there exist free grid points next to every vertex of $G^*$ which guarantees a grid point placement left and right of any vertex $v^*$. During the iteration over the height starting at height 1, every intersection between an edge of $G^*$ and $G$ refer to vertices on the outerface and a new face of $G$ is attached on the outerface, preserving the outerplanarity of the drawing.\\
Since $v_1$ is placed with a distance of $h\cdot r^h$ atop of the root of $G^*$ and $v_2$ and $v_3$ partition the $x$ coordinate of the binary subtrees rooted at the children of the root of $G^*$, planarity is preserved.\\
The total height of the drawing is doubled and the width is tripled compared to a drawing of a $3$-ary tree, therefore the area consumption lies still in $\mathcal{O}(n^2 \log n)$.\\
The construction of $G^*$ with minimal height lies in $\mathcal{O}(n)$ since there are $\mathcal{O}(n)$ faces for any maximal outerplanar graph. Drawing $G^*$ and $G$ lies in $\mathcal{O}(n)$. The runtime of this algorithm is therefore in linear time.\\
The minimal euclidian distance values at least $2^h$ by construction of algorithm \ref{al:k-ary_trees} and lies in $\mathcal{O}(n)$. The length of the longest polyline spans the whole height of the drawing and lies in $\mathcal{O}(n \log n)$. The ratio lies therefore in $\mathcal{O}(h)$ with $h$ describing the height of $G^*$.
\end{proof}
\subsubsection{Example Drawing}

% TODO example drawing of complete outerplanar graph

\subsubsection{Limitations}

As addressed by Observation % TODO ref
, this algorithm works fine for dense weak dual graphs of $G$ since then, the height of $G^*$ is bound by $\mathcal{O}(\log^2 n)$. On the other hand, when $G$ is a loose maximal outerplanar graph, a height of $G^*$ might be of linear size and therefore, the drawing algorithm is not improving the ratio contrary to a straight-line drawing. In addition, the drawing algorithm will only work for simple weak dual graphs. The weak dual graph of a 2-tree is a multigraph, as illustrated by the following small figure:\\
% TODO figure
\\
The approach of drawing the weak dual graph at first followed by the outerplanar graph serves as an idea for a \emph{layered drawing}. The resulting drawings are valid layerings since the vertices of $G$ added on a fixed height of $G^*$ are not adjacent. The layering handles dense outerplanar graphs pretty well since a complete outerplanar graph is drawable with a ratio of $\mathcal{O}(\log n)$. 

\subsection{Drawing Algorithm For Maximal Series Parallel Graphs With Two Bends}

As already observed, a layered drawing algorithm might guarantee a reasonable ratio by defining a minimal distance between two layers and therefore between two adjacent vertices. Since the weak dual graph is not suitable for 2-trees, the \emph{tree decomposition} of an SP-graph will serve as a guidance tool for the sequence of vertices drawn by a layering algorithm. The drawing algorithm will use the \emph{SPQR tree} derived from a tree decomposition.

\subsubsection{Properties Of Maximal Outerplanar Graphs}

% maximal outerplanar stuff

\begin{lemma}\label{l:outerplanar_tree_decomposition}
	Any maximal outerplanar graph $G$ has a tree decomposition $(T,W)$ such that $T$ is of degree at most 3.
\end{lemma}
\begin{proof}
	Consider the weak dual graph $G^*$ considered in Definition \ref{def:complete_maximal_outerplanar}. For vertices $v_1^*,v_2^*$ in $G^*$, insert vertices $t_1,t_2$ in $T$ which are adjacent if $v_1^*,v_2^*$ are adjacent in $G^*$. The corresponding bags $w_1, w_2$ contain the vertices of $G$ which define the face referred by $v_1^*,v_2^*$ in $G^*$. $T$ is isomorphic to $G^*$ and therefore is of degree at most 3.
\end{proof}

% TODO Figure

\begin{lemma}\label{l:outerplanar_two_vertices_subtree_TD_2}
	Let $v,w$ be any two adjacent vertices in a maximal outerplanar graph $G$. Then, the connected subtree $T'$ of a tree decomposition of $G$ containing both $v$ and $w$ contains at most two vertices.
\end{lemma}
\begin{proof}
	If $T'$ would contain at least 3 vertices, then there would exist three distinct vertices in $G$ forming a 3-clique with $v$ and $w$, destroying the outerplanarity property of $G$.
\end{proof}

\begin{lemma}\label{l:outerplanar_TD_properties}
	Let $(T,W)$ be the tree decomposition of a maximal outerplanar graph $G$, $t_1,t_2,t_p \in V(T)$ and $t_1, t_2$ are the children of $t_p$. Then, the following holds:
	\begin{enumerate}
		\item For $i = 1,2$, the bags $w_i$ and $w_p$ share exactly two vertices
		\item $w_1$ and $w_2$ share exactly one vertex
	\end{enumerate}
\end{lemma}
\begin{proof}
	\begin{enumerate}
		\item Since $G$ is a maximal outerplanar graph and therefore a 2-tree, all bags contain exactly 3 vertices. Since $t_p$ is the parent of $t_1$, their bags have two vertices in common when adding a vertex to two adjacent vertices of $t_p$.
		\item This follows directly by Lemma \ref{l:outerplanar_two_vertices_subtree_TD_2}.
	\end{enumerate}
\end{proof}

\begin{lemma}
	Let $(T,W)$ be a tree decomposition of a given maximal outerplanar graph $G$. The subtree $T'$ of $T$ containing a vertex $v$ of $G$ is a list and of length $\mathcal{O}(\texttt{height}(T))$.
\end{lemma}
Let $t_p$ the parent of $t_1,t_2$ and $t_3$ and $v \in w_p$. Assume, that $v \in w_1,w_2,w_3$. Since the bags are of size three, there would be a vertex $v'\in w_p$ such that the subtree of $T$ containing $v,v'$ contains more than two vertices, contradicting Lemma \ref{l:outerplanar_two_vertices_subtree_TD_2}. Then, the subtree $T$ containing $v$ is of degree 2 and therefore a list.

\subsubsection{Properties Of Maximal Series Parallel Graphs}

% 2-tree stuff

\begin{lemma}
	A 2-tree inherits a treewidth of 2.
\end{lemma}
\begin{proof}
	The $K_3$ as a valid 2-tree consists of three vertices and in its tree decomposition $(T,W)$, $T$ consists of one vertex, its bag of the three vertices defining the $K_3$. By adding a vertex $v$ to a 2-tree under the constraint to create a new 3-clique with two already adjacent vertices $v_1,v_2$, a new vertex $t$ is added to $T$ with $\{v,v_1,v_2\}$ in its bag in $W$. Therefore, every bag contains exactly three vertices of $G$ and the treewidth of a 2-tree values 2.
\end{proof}

\begin{lemma}\label{l:2-tree_biconnected}
	Any 2-tree $G$ is biconnected, but not triconnected.
\end{lemma}
This property follows directy by the recursive definition of a SP graph.

\begin{lemma}
	Any $SPQR$-tree $\mathcal{T}_G$ of a 2-tree $G$ consists exclusively of $S,P$ and $Q$ nodes.
\end{lemma}
\begin{proof}[Proof by contradiction]
	If there was an $R$ node in $\mathcal{T}_G$, then $G$ would inherit a triconnected component, contradicting Lemma \ref{l:2-tree_biconnected}.
\end{proof}

\begin{lemma}
	For every 2-tree $G$, there exists a maximal outerplanar subgraph $G'$ that is larger than every other maximal outerplanar subgraph.
\end{lemma}
\begin{proof}
	Consider the tree decomposition $(T_G,W_G)$ of $G$, $T$ inheriting height $h_T$. Add the root of $T$ to $T'$ and its bag to $W'$. From the root of $T$, pick the three children which root the subtrees with the maximum height up to $h_T$ and their respective bags which share exactly one vertex. Then, recursively pick the two children which root the subtrees with the maximum height and add them to $T'$ with their respective bags to $W'$ which share exactly one vertex. This procedure terminates at the leaves of $T$.\\
	Every vertex of the resulting $T'$ is of degree maximum 3 and by Lemma \ref{l:outerplanar_tree_decomposition}, the graph $G'$ induced by $T$ is outerplanar. Since, during this procedure, the nodes with the maximum subtrees are chosen and the properties described in Lemma \ref{l:outerplanar_TD_properties} hold, there exists no other maximal outerplanar subgraph of $G$ which is larger. 
\end{proof}

The tree decomposition of a 2-tree is used to prove the bounds regarding ratio and area consumption for a layered drawing algorithm with two bends. A $SPQR$-tree is derived from a tree decomposition for the implementation of the drawing algorithm.

\begin{lemma}
	There exists a function $f$ which derives an $SPQR$-tree $\mathcal{T}$ from a given tree decomposition $(T,W)$.
\end{lemma}
\begin{proof}
	Every vertex of $T$ refers to a triangle in $G$. Start at the root of $T$. Every $Q$ node refers to an edge of $G$ and is connected to either a $P$ or a $S$ node. A triangle consists of one $P$, one $S$ node and three $Q$ nodes. The following figure demonstrates the possible modifications in a $SPQR$-tree when a vertex is added to an existing 2-tree according to its recursive definition.
	% TODO figure with cases
	In $(T,W)$, any child $t_c$ of $t_p$ is added to the $SQPR$-tree $\mathcal{T}$ accordingly since $w_c$ and $w_t$ share exactly two vertices $v_1,v_2$.
	% TODO cases
\end{proof}

% drawing stuff

\begin{itemize}
	\item definition of active vertex
	\item Shortest subtree first and why
	\item Drawing algorithm according to SPQR
	\item amount of layers according to SPQR per case
\end{itemize}

\subsubsection{Drawing A $2$-Tree With Two Bends}
\newpage
\begin{algorithm}[H]
	\caption{\texttt{DrawSPQR}$(\mathcal{T})$}
	\KwIn{$SPQR$-tree $\mathcal{T}$ of a graph $G$}
	\KwOut{Box drawing $\mathcal{B}_G$}
	Stack \texttt{stack}\\
	\texttt{SPQRVerticesDone} $\gets \emptyset$\\
	$v \gets \mathcal{T}.\texttt{root}$\\
	\tcc{$\mathcal{T}$ is rooted at a $Q$ node with vertice $q_1, q_2$}
	\texttt{stack.push}$(\mathcal{T}.\texttt{root})$\\
	\texttt{column} $\gets 0$\\
	$\delta \gets $ pairwise distance between layers\\
	\texttt{Lower Layer} $l_{low} \gets 0$\\
	\texttt{Upper Layer} $l_{up} \gets l_{low} + \delta$\\
	\tcc{$l_{low}, l_{up}$ describe layers of interest for the remaining drawing}
	$B.\texttt{DrawBox}(q_1, l_{up}, \texttt{column})$\\
	$B.\texttt{DrawBox}(q_2, l_{low}, \texttt{column})$\\
	$\texttt{ActiveVertices} \gets \{q_1,q_2\}$\\
	\While{\texttt{VerticesDone} $\neq V(\mathcal{T})$}{
		$v\gets \texttt{stack.pop}$\\
		List \texttt{children} $\gets v\texttt{.children}$\\
		\texttt{children.SortByHeightOfSubtreesDescending}\\
		\tcc{Every child of a vertex in $\mathcal{T}$ roots a subtree with its respective height}
		\While{\texttt{children} $\neq \emptyset$}{
			\texttt{stack.push(head(children))}
			\texttt{children.remove(head(children))}
		}
		\If{$v$ is $Q$ node}{
			\tcc{A $Q$ node refers to an edge of $G$ between $q_1$ and $q_2$}
			$B$\texttt{.DrawEdge}$((q_1,q_2),\texttt{column})$\\
			\If{$q_1$ unactive}{
				\texttt{ActiveVertices.remove}$(q_1)$
			}
			\If{$q_2$ unactive}{
				\texttt{ActiveVertices.remove}$(q_2)$
			}
			
		}
		\ElseIf{$v$ is $P$ node}{
			$\texttt{column} \gets \texttt{column}+1$\\
			\For{$v\in \texttt{ActiveVertices}$}{
				$B\texttt{.extendBox}$
			}
			$l_{up} \gets layer(p_1)$\\
			$l_{low} \gets layer(p_2)$
		}
		\ElseIf{$v$ is $S$ node}{
			\tcc{Serial composition of vertices $s_1,s_2,s_3$}
			$\texttt{column} \gets \texttt{column}+1$\\
			\If{$\nexists$ free layer between $layer(s_1)$ and $layer(s_3)$}{$l \gets B$\texttt.{addLayer}$(l_{up},l_{low})$}
			\Else{$l \gets $ free layer between $layer(s_1)$ and $layer(s_3)$}
			$B\texttt{.DrawBox}(s_2, l)$\\
			$\texttt{ActiveVertices.add}(s_2)$
		}
		\texttt{SPQRVerticesDone.add}$(v)$
	}
\end{algorithm}

\begin{algorithm}[H]
	\caption{\texttt{Draw2-tree}($G$)}
	\KwIn{2-tree $G$}
	\KwOut{Polyline drawing $\Gamma_G$ with two bends}
	$(T,W) \gets$ tree decomposition of $G$ with lowest height\\
	$(T',W') \gets$ largest maximal outerplanar graph obtained from $(T,W)$\\
	$\mathcal{T'} \gets f(T',W')$\\
	$\texttt{DrawSPQR}(\mathcal{T'})$\\
	% TODO draw difference of 2-tree to largest maximal outerplanar graph
	
\end{algorithm}

\subsubsection{Analysis}

\subsubsection{Example drawing}
