\section*{Abstract}
Graph drawings are diverse in their applications - regarding molecular structure analysis or sensor networks, constructing a drawing with certain properties of distance is of interest. In the year 2022, the \emph{Graph Drawing Symposium} challenges its participants to create small area drawing with maximizing the distances between two adjacent vertices. In a \emph{straight-line drawing}, adjacent vertices are connected by a single line segment. In order to increase the distance between vertices, the vertices are repositioned in a fixed area. In straight-line drawings, the repositioning might be limited in its possibilities. 
Allowing \emph{polylines} loosens this limitation. In a \emph{polyline drawing}, every edge is illustrated as a sequence of line segments which intersect in a point, so-called \textit{bends}. The optimization problem is described as the proportion between the length of the Euclidian distance to the total area consumption. This will be called the \emph{ratio} of a drawing.
\\
On occasion of this Symposium challenge, this thesis will examine the maximization of the \emph{Euclidian distance} on \emph{rooted trees} and small area drawings of the class of \emph{maximal series-parallel graphs}. For a given drawing problem, the class of trees are rather manageable due to their straightforward properties. The graph class of maximal series-parallel graphs is more complex in its properties than rooted trees and investigating the distance maximization of those small area drawings is of particular interest.
\\\\
First, a drawing algorithm for rooted trees is presented which produces straight-line drawings with uniform distances between adjacent vertices.\\\\
Next, the \emph{maximal outerplanar graphs}, a subclass of maximal series-parallel graphs, are investigated for Euclidian distance maximization on small area polyline drawings. It will be proved that they are unbounded, meaning that for every maximal outerplanar graph there exists a larger graph which exceeds the ratio in the respective drawing.\\
A first approach of a polyline drawing algorithm deals with maximal outerplanar graphs of high density. Its limitations will insprire a new idea for dealing with more general occasions. Then, a second approach will serve as a foundation to be extended for the class of maximal series-parallel graphs. An extension of this approach will produce small area polyline drawings for maximal series-parallel graphs with a reasonable ratio.
\\\\
In the extensional work, the previous approaches and results will be applied to a more restrictive class of graphs, called the \emph{planar 3-trees}. It will be demonstrated that the approach for maximal series-parallel graphs is not applicable to the class of 3-trees regarding ratio optimization. 