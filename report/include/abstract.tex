\section*{Abstract}

% Graph drawings are diverse in their applications - considering VLSI or metro map design, \textit{orthogonal} drawings are of interest. In an orthogonal drawing, every edge is illustrated as a sequence of axis-aligned line segments which intersect in a point, so-called \textit{bends}. This thesis will examine one of the most common models of orthogonal drawings - the \textit{Kandinsky model}. The \textit{smoothening} of such Kandinsky drawings introduces 
% First, we will show the possibility of smoothening Kandinsky drawings. Several properties of the drawings are preserved. If the input drawing is crossing-free, the smoothened drawing will also be crossing-free and the orientation of the vertices is not mainly altered. However, the smoothened drawing may increase in size - the horizontal area consumption may be squared in the worst case. The complexity of input Kandinsky drawings may not significantly rise. If the input edge consists of at least four segments, then the complexity of the smoothened edge increases by two.
%Im ersten Abschnitt der Ausarbeitung werden wir zeigen, dass die Glättung einer Kandinsky Zeichnung möglich ist. Ist die Eingangszeichnung kreuzungsfrei, so bleibt diese Eigenschaft erhalten. Die Ausrichtung der Knoten wird dabei nicht grundlegend verändert. Allerdings wird dabei die geglättete Zeichnung größer - der horizontale Platzverbrauch kann sich im schlimmsten Fall quadrieren. Die Komplexität der Kanten bleibt bei ausgehenden Kandinsky Zeichnungen überschaubar. Besteht eine Kante aus mehr als vier Segmenten, erhöht sich dessen Komplexität um zwei. 
% Next, several approaches in order to save segments and area consumption are established. A method is shown, how to shrink the horizontal area consumption by its square root in the best case. A combination of circular arc and line segments is introduced to reduce the area expansion in the smoothened drawing.
%Im nächsten Abschnitt beschäftigen wir uns mit verschiedenen Ansätzen, um an Platzverbrauch und Kantensegmenten der resultierenden Zeichung zu sparen. Wir zeigen einen Ansatz, um den horizontalen Platzverbrauch $b$ im besten Fall um $\sqrt{b}$ zu schrumpfen. Weiter wird eine Kombination aus Kreis- und Liniensegmenten vorgeschlagen, sodass der horizontale Platzverbrauch sich nur um den Wurzelwert $\sqrt{b}$ vervielfacht.
% In the extensional work, a gadget construction is introduced to prove the \textit{\NP-hardness} whether a given graph admits a \textit{bendless} smoothened drawing. \textit{Eighth circular arc segments} - or \textit{octi arcs} - are examined for its properties as they may be relevant for further work.
%In der Erweiterung wird schließlich durch eine Gadgetkonstruktion gezeigt, dass es \textit{\NP-hart }ist zu entscheiden, ob ein Graph mit einer geglätteten Zeichnung ohne Knicke illustriert werden kann. \textit{Achtelkreissegmente} werden auf ihre Eigenschaften untersucht, da sie relevant für weiterführende Projekte sein können.