\section*{Zusammenfassung}
%Firstly, we show that every Kandinsky drawing of arbitrary degree can be postprocessed to a smooth orthogonal drawing and planarity is preserved. As a base for postprocessing a Kandinsky drawing, we use the \textit{Fixed Shape Model}, which preserves the orientation of the vertices. The area bounds of SMOGs are $\Rho(n^2)\times\Rho(n)$ in the worst case. The complexity of a polyedge \textit{does not increase} if and only if the polyedge is \textit{purely uniform} and rise to $\left\lfloor\frac{3}{2}k\right\rfloor$ if and only the the polyedge is \textit{purely alternating}. In practice, it is possible that a polyedge contains several uniform and alternating parts. The \textit{fragmentation} of a polyedge delivers a mathematical description in order to examine all kinds of situations. With help of the optimal fragmentation, we can guarantee that a rather \textit{high complexity} of a polyedge in the original Kandinsky drawing \textit{only increases to} $k+2$.
Graphenzeichnungen sind vielfältig in ihrer Anwendung - im Bezug auf VLSI Design oder Metroliniennetz sind \textit{orthogonale} Zeichnungen besonders relevant. Dies bedeutet, dass jede Kante als Abfolge von horizontalen und vertikalen Liniensegmenten, welche rechtwinklig an sogenannten \textit{Knicken} aneinander knüpfen, dargestellt wird. Ein weitverbreitetes Modell ist das sogenannte \textit{Kandinsky Modell}. Die \textit{Glättung} solcher Zeichnungen arbeitet zusätzlich mit Viertelkreissegmenten, sodass die Ecken abgerundet werden. Dabei ist das Verhalten der sogenannten \textit{Komplexität} der Kanten - aus wievielen Segmenten solch eine Kante in der geglätteten Zeichnung nun besteht - zu untersuchen.
\\\\
Im ersten Abschnitt der Ausarbeitung werden wir zeigen, dass die Glättung einer Kandinsky Zeichnung möglich ist. Ist die Eingangszeichnung kreuzungsfrei, so bleibt diese Eigenschaft erhalten. Die Ausrichtung der Knoten wird dabei nicht grundlegend verändert. Allerdings wird dabei die geglättete Zeichnung größer - der horizontale Platzverbrauch kann sich im schlimmsten Fall quadrieren. Die Komplexität der Kanten bleibt bei ausgehenden Kandinsky Zeichnungen überschaubar. Besteht eine Kante aus mehr als vier Segmenten, erhöht sich deren Komplexität um zwei. 
\\\\
Im nächsten Abschnitt beschäftigen wir uns mit verschiedenen Ansätzen, um an Platzverbrauch und Kantensegmenten der resultierenden Zeichung zu sparen. Wir zeigen einen Ansatz, um den horizontalen Platzverbrauch $b$ im besten Fall auf $\sqrt{b}$ zu schrumpfen. Weiter wird eine Kombination aus Kreis- und Liniensegmenten vorgeschlagen, sodass der horizontale Platzverbrauch sich nur um den Wurzelwert $\sqrt{b}$ vervielfacht.
\\\\
In weiterführender Arbeit wird schließlich durch eine Gadgetkonstruktion gezeigt, dass es \textit{\NP-hart }ist zu entscheiden, ob ein Graph mit einer geglätteten Zeichnung ohne Knicke illustriert werden kann. \textit{Achtelkreissegmente} werden auf ihre Eigenschaften untersucht, da sie relevant für weiterführende Projekte sein können.
%In the next section we figure out several approaches in order to \textit{decrease} the complexity of edges and the area bounds of the smooth orthogonal drawings. A \textit{modified plane sweep} can save up to $\Rho(n)$ area and is able to save some horizontal segments. This plane sweep suits as a handy approach to optimize given SMOG drawings. Further we show an alternative to circular arcs - a \textit{combination} of a quarter arc and a vertical segment increases the complexity, but only needs $\sqrt{r}$ width relative to the original quarter arc.

%Finally, we show a \textit{gadget construction} as a \textit{reduction} from the \NP-hard \textit{SAT problem} to a bendless SMOG with arbitrary degree. \textit{Octi arcs} may be relevant for further work - e.g. illustration of graphs with at most one crossing per edge - and are examined for its properties.