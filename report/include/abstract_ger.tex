\section*{Abstrakt}
Graphenzeichnungen sind vielfach anwendbar - für die Analyse der Molekularstrukturen oder Sensornetzen sind Zeichnungen mit vorbestimmten Distanzen von besonderer Bedeutung. Das \emph{Graph Drawing Symposium} fordert im Jahre 2022 die Teilnehmer des dort stattfindenden Wettbewerbs auf, Zeichnungen auf kleiner Fläche anzufertigen und dabei die Distanzen der paarweise verbundenen Knoten zu maximieren. In \emph{geradlinigen Zeichnungen} werden Knoten mit einem einzigen Liniensegment verbunden. Um die jeweiligen Distanzen zwischen Knoten zu maximieren, ist eine Umpositionierung der Knoten erforderlich. Dies erweist sich bei geradlinigen Zeichnungen als bedingt möglich. Lässt man \emph{Mehrfachlinien} zu, erweitert dies die Möglichkeiten der Knotenrepositionierung. Bei einer \emph{Mehrfachlinienzeichnung} wird eine Verbindung zwischen zwei Knoten als eine Sequenz von Liniensegmenten, welche sich in einem Punkt schneiden, dargestellt. Dieser Schnittpunkt wird im Allgemeinen \emph{Knick} genannt. Das Optimierungsproblem wird als das Verhältnis zwischen der kürzesten euklidischen Distanz zwischen zwei verbundenen Knoten und dem gesamten Flächenverbrauch beschrieben. Dieses Verhältnis wird \emph{Ratio} genannt.
\\\\
Anlässlich der diesjährigen Symposium Challenge untersucht dieses Ausarbeitung die Maximierung der euklidischen Distanz für die Klasse der \emph{verwurzelten Bäume} und der \emph{maximalen seriell-parallelen Graphen}. Im Allgemeinen ist die Klasse der Bäume durch ihre überschaubaren Eigenschaften eine gute erste Anlaufstelle zur Untersuchung eines Optimierungsproblems. Da die Klasse der maximal seriell-parallelen Graphen um einiges vielseitiger in ihren Eigenschaften ist, ist die Optimierung von dessen Zeichnungen mit geringem Platzverbrauch von hohem Interesse.
\\\\
Zuerst wird ein Zeichenalgorithmus für verwurzelte Bäume präsentiert, welcher geradlinige Zeichnungen mit einer optimalen Ratio erstellt.\\\\
Darauffolgend wird eine Subklasse der maximal seriell-parallelen Graphen, die sogenannten \emph{maximal outerplanaren} Graphen untersucht. Es wird gezeigt, dass die Ratio \emph{unbeschränkt} ist, was heißt, dass es für jeden maximal outerplanaren Graphen einen größeren gibt, der das Ratio der dementsprechenden Zeichnung signifikant erhöht.\\
Ein erster Ansatz für eine Mehrfachlinienzeichnung setzt sich mit maximal outerplanaren Graphen hoher Dichte auseinander. Die Begrenzungen dieses Ansatzes dienen zur Inspiration für einen zweiten, allgemeineren Ansatz. Dieser zweite Ansatz dient als eine Grundlage für die Erstellung von Mehrfachlinienzeichnungen von maximal seriell-parallelen Graphen. Die Ansatzerweiterung resultiert in einem Zeichenalgorithmus, welcher Mehrfachlinienzeichnungen von maximalen seriell-parallelen Graphen auf vertretbar kleiner Fläche produziert, sodass die Ratio in einem angemessenen Rahmen bleibt.\\\\
Zusätzlich wurde eben dieser Ansatz auf eine restriktivere Klasse an Graphen, den sogenannten \emph{3-Trees}, angewandt. Es wird illustriert, dass dieser Ansatz bezüglich der Ratiooptimierung für die Klasse der 3-Trees unzureichend ist.