\section{Introduction}

The topic of visualization of information relationships occur in various areas of work. Examples of the fields include circuit design, architecture, web science, social sciences, biology, geography, information security and software engineering.
% TODO cites for this examples
Over the last decades, many different efficient algorithms were developed for graph drawings in the Euclidean plane.\\
Different quality measures for graph drawings have been considered, including area consumption, angular resolution, slope number, average edge length, and total edge length, addressing the readability and aesthetics \cite{edge-length-ratio-2tree}.

\bigskip
In context of this thesis, drawings with prescribed edge lengths is of particular interest. Its relevance addresses aspects of computational geometry, rigidity theory, structural analysis of molecules and sensor networks \cite[P. 1]{straight-line_2-trees}. In general, it is hard to decide \cite[P. 2]{straight-line_2-trees} whether a given graph admits a drawing with prescribed edge lengths under the condition that any edge is drawn with a straight-line and no two edges cross each other, even in the case of uniform lengths for all edges.
% Readability

\bigskip

% 3. Symposium

Starting from a workshop in 1994, the first international conference for \emph{Graph Drawing} was held in Passau in 1995 \cite{GD:Symposium}. The annual symposium covers topics of combinatorical and algorithmic aspects of graph drawing as well as the design of network visualization systems and interfaces.\\
One part of the symposium is the \emph{Graph Drawing Contest}. The contest consists of two parts - the \emph{Creative Topics} and the \emph{Live Challenge}. The main focus for the Creative Topics lies on the creation of drawings of two given graphs. Aspects to consider for the visualization are clarity, aesthetic appeal and readability.\\
On the other hand, the Live Challenge is held similar to a programming contest. Participants, usually teams, will get a theme and a set of graphs and will have one hour of processing. The results will be ranked and the team with highest score wins the competition. The teams will be allowed to use any combination of software and human interaction systems in order to produce the best results. Usually, the challenge is derived from a theoretical optimization problem \cite{GD:2021}.

\bigskip

In 2021, the Live Challenge during the 29th International Symposium on Graph Drawing and Network Visualization held in Tübingen, Germany addressed the optimization of graph drawing edge lengths. An \emph{edge length ratio} of a drawing describes the proportion between the \emph{minimal and maximal edge lengths}. The size of the total area of a drawing affects the maximum edge length. When considering \emph{straight-line drawings}, where edges are a single straight line segment, the ratio scales in proportion of the total area size.\\
For the \emph{Live Challenge}, the goal was to produce a \emph{polyline graph drawing}, where edges are line segments joined together, with uniform edge lengths. 
The difficulty of this challenge was intensified by constraints on the drawing area and the amount of line segments per edge \cite{GD:2021_Challenge}.

\bigskip

% 2. Bring in euclidian distances in drawing

In 2022, the 30th International Symposium of Graph Drawing and Network Visualization held in Tokio, Japan \cite{GD:2022_Challenge} addresses an alternation of the Live Challenge from previous year. In contrast to the edge length ratio from 2021, this years ratio describes the proportion of the \emph{maximal polyline edge length} to the \emph{minimal Euclidian distance} between two adjacent vertices. 

\bigskip

% 3. What is this Thesis about

This thesis will address the topic of minimizing the difference between the longest and shortest edge length for certain graph classes. The main goal is to maximize the distances between two adjacent vertices while keeping the area consumption of the resulting drawing at a low level. Allowing to reroute edges through so-called \emph{bend points} in a drawing without causing crossings between edges increases the flexibility of finding an approach for the edge length difference minimization approach.\\
In Section \ref{section:preliminaries}, the preliminaries and terminology are defined. 
In Section \ref{section:initial_situation}, the general problem considering the edge length ratio is formalized. Furthermore, the potential for ratio improvement is illustrated by allowing polyline edges. 
Section \ref{s:k-ary_trees} describes a drawing algorithm for the graph class of \emph{$k$-ary trees} which guarantees a satisfying edge length ratio.
Section \ref{section:SP-graphs} contains drawing algorithms for the graph class of \emph{series parallel graphs}. The subclass of \emph{outerplanar} graphs and \emph{$2$-trees} are of particular interest. Those drawings will improve the worst case ratio behaviour described in Section \ref{section:initial_situation}.
In section \ref{section:related_work}, work related to the content of this thesis will be presented and section \ref{section:future_work} describes future work.