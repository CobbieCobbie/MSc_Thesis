\section{$k$-ary Trees}
% TODO What is this section about?

% properties of complete k-ary trees

% k is a constant

% context for outerplanar graphs

% The height 

\begin{lemma}The \emph{height $h$} of a complete $k$-ary tree $T$ is in $\mathcal{O}(\log n)$.
\end{lemma}
\begin{proof}
	\begin{align}
		n = \sum_{i=0}^{h}k^i &= \frac{k^{h+1}-1}{k-1}\\
		\Leftrightarrow h &= \log_k((k-1)n+1)-1\\
		&= \frac{\log((k-1)n+1)}{\log k}-1\\
		\Rightarrow h &\in O(\log n)
	\end{align}
\end{proof}

% Drawing result

\begin{theorem}
	Every $k$-ary tree admits a planar straight-line drawing with a nearly optimal ratio, apart from a rounding error, on area $\mathcal{O}(n^2\log n)$.
\end{theorem}
\begin{proof}
	The following drawing will be constructed from top to bottom, meaning that the $y$-coordinates of the children of any vertex $v$ are smaller than the $y$ coordinate of $v$.\\
	Let $r := k^h$. For a vertex $v$ in height $i$, consider $k$ equidistant columns with $x$-coordinates between $x(v) - k^{h-i}$ and $x(v) + k^{h-i}$. These are integer coordinates since the distance between two columns next to each other equals $2\cdot \frac{k^{h-i}}{k}$. Draw a circle around $v$ with radius $r$. Choose the grid points $v_i$ on the column closest to the resulting intersections with the constraint that $y(v_i) \leq y(v)$ and connect $v$ with its $k$ children with a straight-line.
	% TODO Illustration of circle with columns
	\\
	The distance between two neighbouring columns in height $i$ suffices for the remaining drawing since it holds for the remaining heights:
	\begin{align}
		\underbrace{2\cdot \sum_{j=i}^{h-1} k^{h-j-1}}_{\text{drawn from both columns}} &= 2\cdot\sum_{z = 0}^{h-i-1}h^z\\
		&= 2\cdot\frac{k^{h-i}-1}{k-1} < 2\cdot k^{h-i}
	\end{align}
	The following algorithm sums up the approach described above.\\
	\begin{algorithm}[H]
		\KwIn{complete $k$-ary tree $T$,$k$}
		\KwOut{Planar drawing of $T$ with nearly optimal ratio}
		\caption{Drawing algorithm for $k$-ary trees}
		$h \gets $ \text{height of $T$}\\
		$r \gets k^h$\\
		Draw $root(T)$ on any grid point\\
		\texttt{Draw}$(root(T),r,h)$
	\end{algorithm}
	\begin{algorithm}[H]
		\KwIn{Already drawn vertex $v$, radius and height $r,h \in \mathbb{N}$}
		\KwOut{Coordinates of all the children of $v$}
		\caption{\texttt{Draw}$(v,r,h)$}
		\If{$v$ leaf}{\Return}
		\Else{
			$h' \gets \texttt{height}(v)$\\
			$d \gets 2\cdot k^{h-h'-1}$\\
			$C \gets v.\texttt{DrawCircle}(r)$\\
			\tcc{Draw circle with radius $r$ around $v$}
			\For{$i \in [1..k]$}{
				$x(v_i) \gets x(v) - k^{h-h'} + (i-1)\cdot d$\\
				\tcc{$x$-coordinate of $i$-th child of $v$}
				$X \gets \texttt{Column}(x(v_i))$\\
				\tcc{Identify the column at position $x(v_i)$}
				$s \gets \texttt{Intersection}(I,X)$\\
				\tcc{Calculate the intersection of the circle $C$ and the column $X$}
				$y(v_i) \gets round(y(s))$\\
				\texttt{DrawStraightLine}$(v,v_i)$\\
				\texttt{Draw}$(v_i,r,h)$
			}
		}		
	\end{algorithm}
	The height of the drawing is bound by $h\cdot r = h\cdot k^h \in \mathcal{O}(n\cdot \log n)$. 
	The width is bound by $2\cdot\sum_{i = 0}^{h} k^i = 2\cdot \frac{k^{i+1}-1}{k-1} \in \mathcal{O}(n)$, resulting in $\mathcal{O}(n^2\log n)$ area.\\ 
	Since the algorithm works from top to bottom and for height $i$, the area for every subtree is disjointedly reserved, the resulting drawing is planar. Furthermore, all straight-line edges inherit a length of approximately $k^h$, no bends were used and the ratio is bound by $1+\varepsilon, 0\leq \varepsilon<1$.
	
\end{proof}

% TODO Example drawings
