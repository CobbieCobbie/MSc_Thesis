\section{Future Work}
\subsubsection*{Further use of the fragmentation}
The main purpose of the fragmentation is having a subdivision of complex polyedges. In the orthogonal case, we are able to distinguish uniform parts from alternating parts in a large polyedge. It is possible to extend the work of field with similar results to other classes of drawings.
\subsubsection*{Implementation}
Now that we achieved the guarantee of Kandinsky drawings being able to be postprocessed to a SMOG, an \textit{implementation} - for example with the \texttt{yFiles}-bibliography in \textit{Java} - would be of interest. If we got the implementation, then we would be able to examine a set of orthogonal drawings and their smoothened results - the appearance of Kandinsky drawings and SMOGs. Further, we would be able to make a statement whether an area saving measure like the circular arc substitution would result in visibly clear drawings. 
\subsubsection*{Graphs with crossings}
The results in this work consider only graphs with planar drawings. It would be desirable to consider graph with \textit{non-planar drawings}. The illustration of crossings shall be visibly distinguishable from vertices and polyedges. Our first approach of introducing octi arcs did result in a non-consistent model since octi arcs with a 45\degree~bend introduce new slopes. 
\subsubsection*{Further saving approaches}
Certain properties of a polyedge fragmentation, circular arc substitution, the saving plane sweep and the port reassignment enables us to save some \textit{bends} and \textit{area consumption} of the drawing. Regarding the port reassignment, we considered alternating polyedges of complexity three. Naturally, it would be desireable to consider the port reassignment in a more general case. Also a point of interest would be saving measures by cutting horizontally \underline{and} vertically. So the choice of cut alignment may result in an even better drawing. Cutting through a circular arc could mean a substitution with a smaller circular arc and a line segment accordingly.