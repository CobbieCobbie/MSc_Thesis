\section{Preliminaries}
\subsection{Definitions}
%A graph is a 2-tuple consisting of a vertex set and an edge set. The visualization, however, has to be drawn in some kind of way. Investigating the drawing of a graph needs to include several constraints, for example, \emph{how} we want the drawing and \emph{where} we want to draw on. These constraints can be described mathematically.\\
As otherwise mentioned, a \emph{graph} $G=(V,E)$ is a tuple consisting of two sets - the set of vertices and the set of edges. An \emph{edge} $e = (v,w), v,w \in V$ is a tuple and describes a connectivity relation between two vertices.
% Undirected
Unless otherwise mentioned, the graphs are \emph{undirected}, meaning that the edge $(u,v)$ is identical to the edge $(v,u)$.
% Face
A \emph{face} is a maximal open region of the plane bounded by edges. 
% degree of G
The \emph{degree} of a vertex states the amount of edges incident to the vertex.
%  The \emph{degree} of a graph $G$ is the maximum of the degree of its vertices. 
% Grid

% Embedding
An \emph{embedding} of $G$ is the collection of counter-clockwise circular orderings of edges around each vertex of $V$.
% Drawing
A \emph{drawing} $\Gamma$ of a graph $G$ is a function, where each vertex is mapped on a unique point $\Gamma(v)$ in the plane and each edge is mapped on an open Jordan curve $\Gamma(e)$ ending in its vertices.
% Planarity
A graph is \emph{planar} if and only if there exists a crossing-free representation in the plane. 
\cite[Page 100]{DBLP:books/daglib/0023376}
% Area

% Straight-line drawing

% Box drawing

% Poly line drawing

% Euclidian distance

% Edge length

% ratio

\subsection{Graph Classes}
% tree

% k-nary tree

% series parallel graph

% outerplanar graph

% 2-tree

\subsection{Tools}

% SPQR Tree
