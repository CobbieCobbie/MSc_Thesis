\section{Preliminaries}
\subsection{Definitions}
%A graph is a 2-tuple consisting of a vertex set and an edge set. The visualization, however, has to be drawn in some kind of way. Investigating the drawing of a graph needs to include several constraints, for example, \textit{how} we want the drawing and \textit{where} we want to draw on. These constraints can be described mathematically.\\
As otherwise mentioned, a \textit{graph} $G=(V,E)$ is a tuple consisting of two sets - the set of vertices and the set of edges. An \textit{edge} $e = (v,w), v,w \in V$ is a tuple and describes a connectivity relation between two vertices. Unless otherwise mentioned, the graphs are \textit{undirected}, meaning that the edge $(u,v)$ is identical to the edge $(v,u)$. 
A \textit{face} is a maximal open region of the plane bounded by edges. 
The \textit{degree} of a vertex states the amount of edges incident to the vertex. The \textit{degree} of a graph $G$ is the maximum of the degree of its vertices. 
% Drawing
An \textit{embedding} of $G$ is the collection of counter-clockwise circular orderings of edges around each vertex of $V$.
A \textit{drawing} $\Gamma$ of a graph $G$ is a function, where each vertex is mapped on a unique point $\Gamma(v)$ in the plane and each edge is mapped on an open Jordan curve $\Gamma(e)$ ending in its vertices. 
A graph is \textit{planar} if and only if there exists a crossing-free representation in the plane. 
\cite[Page 100]{DBLP:books/daglib/0023376}