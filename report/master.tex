\documentclass[a4paper,12pt,headsepline]{scrartcl}

%\part{title}
\usepackage[utf8]{inputenc}
\usepackage{graphicx}
\usepackage{caption,subcaption}
\usepackage[english]{babel}
\usepackage[T1]{fontenc}
\usepackage{geometry}
\usepackage{proof}
\geometry{left=3.5cm, right=2cm, top=2.5cm, bottom=2cm}
\usepackage{hyperref}
%\usepackage[hyphens,obeyspaces,spaces]{url}
\usepackage{fancybox}
\usepackage{amssymb,amsmath,amsthm}
\usepackage{gensymb}
\usepackage[linesnumbered,ruled,vlined,norelsize]{algorithm2e}
%\usepackage[bookmarksnumbered,pdftitle={\titleDocument},hyperfootnotes=false]{hyperref} 
\usepackage{color}
\usepackage{float}

%test
\usepackage[backend=bibtex]{biblatex}
\usepackage{filecontents}

\addbibresource{ref.bib}

\restylefloat{figure}

% Makros
\newenvironment{sketch}{\begin{proof}[Proof (Sketch)]}{\end{proof}}
\newtheorem{theorem}{Theorem}
\newtheorem{assumption}{Assumption}
\newtheorem{lemma}{Lemma}
\newtheorem{remark}{Remark}
\newtheorem{definition}{Definition}
\newtheorem{corollary}{Corollary}
\newcommand{\comment}[1]
{
  \begin{quotation}
    \textcolor{blue}{\underline{Edit:} #1}
  \end{quotation}
}
\newcommand{\TODO}[1]
{
  \begin{quotation}
    \textcolor{red}{\underline{TODO:} #1}
  \end{quotation}
}
\newcommand{\mygraphics}[3][2]{
\begin{figure}[h]
	\centering
	\includegraphics[page=#1]{#2}
	\caption{#3}
\end{figure}
}

% Zeichen 
\newcommand{\Rho}{\ensuremath{\mathcal{O}}}
\newcommand{\ec}{\texttt{ec}}
\newcommand{\NP}{\call{NP}}
\newcommand{\call}[1]{\ensuremath{\mathcal{#1}}}

% neue Kopfzeilen mit fancypaket
\usepackage{fancyhdr} %Paket laden
\pagestyle{fancy} %eigener Seitenstil
\fancyhf{} %alle Kopf- und Fußzeilenfelder bereinigen
\fancyhead[L]{\nouppercase{\leftmark}} %Kopfzeile links
\fancyhead[C]{} %zentrierte Kopfzeile
\fancyhead[R]{\thepage} %Kopfzeile rechts
\renewcommand{\headrulewidth}{0.4pt} %obere Trennlinie
%\fancyfoot[C]{\thepage} %Seitennummer
%\renewcommand{\footrulewidth}{0.4pt} %untere Trennlinie

\frenchspacing
\makeindex

% Pseudocode für Java
\usepackage{listings}
\lstset{numbers=left, numberstyle=\tiny, numbersep=5pt, keywordstyle=\color{black}\bfseries, stringstyle=\ttfamily,showstringspaces=false,basicstyle=\footnotesize,captionpos=b}
\lstset{language=java}

% Disable single lines at the start of a paragraph (Schusterjungen)
\clubpenalty = 10000
% Disable single lines at the end of a paragraph (Hurenkinder)
\widowpenalty = 10000
\displaywidowpenalty = 10000

\begin{document}
%\newpage
%\thispagestyle{empty} % erzeugt Seite ohne Kopf- / Fusszeile
%\section*{ }
%Seitennummerierung römisch
\pagenumbering{roman} 
% das Papierformat zuerst
%\documentclass[a4paper, 11pt]{article}

% deutsche Silbentrennung
%\usepackage[ngerman]{babel}

% wegen deutschen Umlauten
%\usepackage[ansinew]{inputenc}

% hier beginnt das Dokument
%\begin{document}


\thispagestyle{empty}

%\begin{figure}[t]
% \centering
% \includegraphics[width=0.6\textwidth]{abb/logo1}
%~~~~~~~~~~
% \includegraphics[width=0.3\textwidth]{abb/logo2}
%\end{figure}


\begin{verbatim}


\end{verbatim}

\begin{center}
\Large{Eberhard Karls Universität Tübingen}\\
\small Wilhelm Schickard Institut Tübingen\\
\end{center}


\begin{center}
\Large{Fachbereich Informatik}
\end{center}
\begin{verbatim}




\end{verbatim}
\begin{center}
%\doublespacing
\textbf{\large{On Maximizing The Euclidian Distance Between Vertices In Drawings Of Series Parallel Graphs}}\\
%\singlespacing
\begin{verbatim}

\end{verbatim}
\textbf{{Arbeitsbereich Algorithmik}}
\end{center}
\begin{verbatim}

\end{verbatim}
\begin{center}

\end{center}
\begin{verbatim}

\end{verbatim}
\begin{center}
\textbf{zur Erlangung des akademischen Grades \\ Master of Science}
\end{center}
\begin{verbatim}






\end{verbatim}
\begin{flushleft}
\begin{tabular}{llll}
\textbf{Autor:} & & Benjamin \c Coban & \\
& & MatNr. 3526251 & \\
& & \\
\textbf{Version vom:} & & \foreignlanguage{ngerman}{\myformat\today} &\\
& & \\
\textbf{ErstprüferIn:} & & Prof. Dr. Michael Kaufmann &\\
\textbf{ZweitprüferIn:} & & Prof. Dr. Ulrike von Luxburg &\\
\end{tabular}
\end{flushleft}
\section*{Zusammenfassung}
Graphenzeichnungen sind vielfach anwendbar - für die Analyse der Molekularstrukturen oder Sensornetzen sind Zeichnungen mit vorbestimmten Distanzen von besonderer Bedeutung. Das \emph{Graph Drawing Symposium} fordert im Jahre 2022 die Teilnehmer des dort stattfindenden Wettbewerbs auf, Zeichnungen auf kleiner Fläche anzufertigen und dabei die Distanzen der paarweise verbundenen Knoten zu maximieren. In \emph{geradlinigen Zeichnungen} werden Knoten mit einem einzigen Liniensegment verbunden. Um die jeweiligen Distanzen zwischen Knoten zu maximieren, ist eine Umpositionierung der Knoten erforderlich. Dies erweist sich bei geradlinigen Zeichnungen als bedingt möglich. Lässt man \emph{Polygonzüge} zu, erweitert dies die Möglichkeiten der Knotenrepositionierung. Bei einer \emph{Polygonzugzeichnung} wird eine Verbindung zwischen zwei Knoten als eine Sequenz von Liniensegmenten mit unterschiedlicher Steigung, welche sich in einem Punkt schneiden, dargestellt. Dieser Schnittpunkt wird im Allgemeinen \emph{Knick} genannt. Die Optimierungsfunktion wird als das Verhältnis zwischen der kürzesten euklidischen Distanz zwischen zwei verbundenen Knoten und dem gesamten Flächenverbrauch beschrieben. Dieses Verhältnis wird die \emph{Ratio} genannt.
\\\\
Anlässlich der diesjährigen Symposium Challenge untersucht dieses Ausarbeitung die Maximierung der euklidischen Distanz für die Klasse der \emph{verwurzelten Bäume} und der \emph{maximalen seriell-parallelen Graphen}. Im Allgemeinen ist die Klasse der Bäume durch ihre überschaubaren Eigenschaften eine gute erste Anlaufstelle zur Untersuchung eines Optimierungsproblems. Da die Klasse der maximal seriell-parallelen Graphen um einiges vielseitiger in ihren Eigenschaften ist, ist die Optimierung von dessen Zeichnungen mit geringem Platzverbrauch von hohem Interesse.
\\\\
Zuerst wird ein Zeichenalgorithmus für verwurzelte Bäume präsentiert, welcher geradlinige Zeichnungen mit einer optimalen Ratio erstellt.\\\\
Darauffolgend wird eine Subklasse der maximal seriell-parallelen Graphen, die sogenannten \emph{maximal outerplanaren} Graphen untersucht. Es wird gezeigt, dass die Ratio \emph{unbeschränkt} ist, was heißt, dass es für jeden maximal outerplanaren Graphen einen größeren gibt, der das Ratio der dementsprechenden Zeichnung signifikant erhöht.\\
Ein erster Ansatz für eine Polygonzugzeichnung setzt sich mit maximal outerplanaren Graphen hoher Dichte auseinander. Die Begrenzungen dieses Ansatzes dienen zur Inspiration für einen zweiten, allgemeineren Ansatz. Dieser zweite Ansatz dient als eine Grundlage für die Erstellung von Polygonzugzeichnungen von maximal seriell-parallelen Graphen. Die Ansatzerweiterung resultiert in einem Zeichenalgorithmus, welcher Polygonzugzeichnungen von maximalen seriell-parallelen Graphen auf vertretbar kleiner Fläche produziert, sodass die Ratio in einem angemessenen Rahmen bleibt.\\\\
Zusätzlich wurde eben dieser Ansatz auf eine restriktivere Klasse an kreuzungsfreien Graphen, den sogenannten \emph{planaren 3-Trees}, angewandt. Es wird illustriert, dass dieser Ansatz bezüglich der Ratiooptimierung für die Klasse der 3-Trees unzureichend ist.
\section*{Abstract}
Graph drawings are diverse in their applications - regarding molecular structure analysis or sensor networks, constructing a drawing with certain properties of distance is of interest. In the year 2022, the \emph{Graph Drawing Symposium} challenges its participants to create small area drawing with maximizing the distances between two adjacent vertices. In a \emph{straight-line drawing}, adjacent vertices are connected by a single line segment. In order to increase the distance between vertices, the vertices are repositioned in a fixed area. In straight-line drawings, the repositioning might be limited in its possibilities. 
Allowing \emph{polylines} loosens this limitation. In a \emph{polyline drawing}, every edge is illustrated as a sequence of line segments which intersect in a point, so-called \textit{bends}. The optimization problem is described as the proportion between the length of the Euclidian distance to the total area consumption. This will be called the \emph{ratio} of a drawing.
\\
On occasion of this Symposium challenge, this thesis will examine the maximization of the \emph{Euclidian distance} on \emph{rooted trees} and small area drawings of the class of \emph{maximal series-parallel graphs}. For a given drawing problem, the class of trees are rather manageable due to their straightforward properties. The graph class of maximal series-parallel graphs is more complex in its properties than rooted trees and investigating the distance maximization of those small area drawings is of particular interest.
\\\\
First, a drawing algorithm for rooted trees is presented which produces straight-line drawings with uniform distances between adjacent vertices.\\\\
Next, the \emph{maximal outerplanar graphs}, a subclass of maximal series-parallel graphs, are investigated for Euclidian distance maximization on small area polyline drawings. It will be proved that they are unbounded, meaning that for every maximal outerplanar graph there exists a larger graph which exceeds the ratio in the respective drawing.\\
A first approach of a polyline drawing algorithm deals with maximal outerplanar graphs of high density. Its limitations will insprire a new idea for dealing with more general occasions. Then, a second approach will serve as a foundation to be extended for the class of maximal series-parallel graphs. An extension of this approach will produce small area polyline drawings for maximal series-parallel graphs with a reasonable ratio.
\\\\
In the extensional work, the previous approaches and results will be applied to a more restrictive class of graphs, called the \emph{planar 3-trees}. It will be demonstrated that the approach for maximal series-parallel graphs is not applicable to the class of 3-trees regarding ratio optimization. 
\section*{Erklärung}
Hiermit erkläre ich, dass ich diese schriftliche Abschlussarbeit selbstständig verfasst habe, keine anderen als die angegebenen Hilfsmittel und Quellen benutzt habe und alle wörtlich oder sinngemäß aus andern Werken übernommenen Aussagen als solche gekennzeichnet habe.
\begin{verbatim}




































\end{verbatim}
\begin{flushright}
	------------------------------------------------------------\\
	\small Datum, Ort, Unterschrift
\end{flushright}
\tableofcontents%neue Seitennummerierung, arabisch, reset
\newpage
\pagenumbering{arabic}
\setcounter{page}{1}
\section{Introduction}

Thoughts:
\begin{enumerate}
	\item General graph drawing introduction
	\item Graph Drawing Symposium introduction
	\item why is maximizing distances between vertices of interest?
	\item Figures of readability
	\item Thesis structure
\end{enumerate}

% 1.


The topic of visualization of information relationships occur in various areas of work. Examples of the fields include circuit design, architecture, web science, social sciences, biology, geography, information security and software engineering. Over the last decades, many different efficient algorithms were developed for graph drawings in the Euclidean plane.\\
Different quality measures for graph drawings have been considered, including area, angular resolution, slope number, average edge length, and total edge length \cite{edge-length-ratio-2tree}, addressing the readability and aesthetics 
% Readability
\bigskip

% 3. Symposium

Starting from a workshop in 1994, the first international conference for \emph{Graph Drawing} was held in Passau in 1995 \cite{GD:Symposium}. The annual symposium covers topics of combinatorical and algorithmic aspects of graph drawing as well as the design of network visualization systems and interfaces.\\
One part of the symposium is the \emph{Graph Drawing Contest}. The contest consists of two parts - the \emph{Creative Topics} and the \emph{Live Challenge}. The main focus for the Creative Topics lies on the creation of drawings of two given graphs. Aspects to consider for the visualization are clarity, aesthetic appeal and readability.\\
On the other hand, the Live Challenge is held similar to a programming contest. Participants, usually teams, will get a theme and a set of graphs and will have one hour of processing. The results will be ranked and the team with highest score wins the competition. The teams will be allowed to use any combination of software and human interaction systems in order to produce the best results. Usually, the challenge is derived from a theoretical optimization problem \cite{GD:2021}.

\bigskip

In 2021, the Live Challenge during the 29th International Symposium on Graph Drawing and Network Visualization held in Tübingen, Germany addressed the optimization of graph drawing edge lengths. An \emph{edge length ratio} of a drawing describes the proportion between the \emph{minimal and maximal edge lengths}. The size of the total area of a drawing affects the maximum edge length. When considering \emph{straight line drawings}, where edges are a single straight line segment, the ratio scales in proportion of the total area size.\\
For the \emph{Live Challenge}, the goal was to produce a \emph{polyline graph drawing}, where edges are line segments joined together, with uniform edge lengths. 
The difficulty of this challenge was intensified by constraints on the drawing area and the amount of line segments per edge \cite{GD:2021_Challenge}.

\bigskip

% 2. Bring in euclidian distances in drawing

In 2022, the 30th International Symposium of Graph Drawing and Network Visualization held in Tokio, Japan \cite{GD:2022_Challenge} addresses an alternation of the Live Challenge from previous year. In contrast to the edge length ratio from 2021, this years ratio describes the proportion of the \emph{maximal polyline edge length} to the \emph{minimal Euclidian distance} between two adjacent vertices. 

\bigskip

% 3. What is this Thesis about

This thesis contains the examination of maximization of the \emph{Euclidian distance} between two adjacent vertices in small area drawings of certain graph classes.
In Section 1, the preliminaries and terminology are defined. 
In Section 2, the general problem considering the edge length ratio is formalized. Furthermore, the potential for ratio improvement is illustrated by allowing polyline edges. 
Section 3 describes a drawing algorithm for the graph class of \emph{$k$-ary trees} which guarantees a satisfying edge length ratio.
Section 4 contains drawing algorithms for the graph class of \emph{series parallel graphs}. The subclass of \emph{outerplanar} graphs and \emph{$2$-trees} are of particular interest. Those drawings will improve the worst case ratio behaviour described in Section 2.
Section 5 

% Research Project



%\section{Introduction}
%The topic of visualization of information relationships occur in various areas of work. Examples of the fields include circuit design, architecture, web science, social sciences, biology, geography, information security and software engineering. Over the last decades, many different efficient algorithms were developed for graph drawings in the Euclidean plane.\\

%Among others, one classic question is to test whether a given network can be visualized with straight lines and prescribed edge lengths. This study is also related to several other topics like rigidity theory, structural analysis of molecules and sensor networks [\cite{DBLP:journals/corr/abs-2108-12628}, Page 1].

%For over 25 years, an international symposium of Graph Drawing and Network Visualization takes place annually. $28^{\text{th}}$ International Symposium of Graph Drawing and Network Visualization will be held from September $15^{\text{th}}$ to $17^{\text{th}}$ in Tübingen. 

%\newline Part of the symposium is a traditional Graph Drawing Contest. The contest consists of two parts - the \textit{Creative Topics} and the \textit{Live Challenge}. The main focus for the Creative Topics lies on the creation of drawings of two given graphs. Aspects to consider for the visualization are clarity, aesthetic appeal and readability.
% \newline On the other hand, the Live Challenge is held similar to a programming contest. Participants, usually teams, will get a theme and a set of graphs and will have one hour of processing. The results will be ranked and the team with the highest score wins the competition. The teams will be allowed to use any combination of software and human interaction systems in order to produce the best results. Usually, the challenge is derived from a theoretical optimization problem.\bigskip\\

% On the occasion of the Graph Drawing contest held this year in Tübingen, this report covers the topic of drawing various graph classes with connections of approximately equal length.



% BSc Thesis

%Metro maps, circuits, networks, construction plans and many more - they all can be visualized with a corresponding \textit{graph drawing}. Over the last decades, many different efficient algorithms were developed for graph drawings in the Euclidean plane. Especially, orthogonal graph drawings are of interest as they are applicable in various fields.
%\\In order to work with graph drawings efficiently, one has to consider the \textit{quality} of a graph drawing. There is a huge variety of aspects to consider when we want to examine the quality of a drawing. The readability of the illustrated information, the size of the drawing - measured with the pair of vertices with the farthest distance in the drawing, and the \textit{edge complexity} - the amount of consecutive line segments for an edge illustration are only a few aspects how to measure the drawing quality. Naturally, we try to create drawings as clearly as possible meaning to avoid drawings with a high edge complexity. If a given graph admits a \textit{crossing-free}, or in other words \textit{planar} drawing, we want to preserve this property in further processing approaches.
%\\The American abstract artist \textit{Mark Lombardi} gained approval for his aesthetic illustration of political-economic structures. The diagrams included \textit{circular arcs} of different sizes and their even distribution around a vertex in order to vizualize connections adequately. It seems that the circular arcs emphasize the connection between components in sense of direction.
%\begin{figure}[H]
%	\centering
%	\begin{subfigure}{\textwidth}
	%		\centering
	%		\includegraphics[width=0.8\linewidth]{includegraphics/Introduction_Lombardi-example}
	%	\end{subfigure}
%\caption{Work of Mark Lombardi \cite{lombardi_ex}}\label{im:lombardi_ex}
%\end{figure}
%\textit{Orthogonal drawings} arise among others in VLSI design where quite many cables are following a similar path. The smallest angle between axis-aligned line segment is at most $\pi/2$ and their angular resolution is quite pleasing for the eye of the viewer. One fundamental, reliable model is the \textit{Kandinsky model} which is based on a \textit{grid embedding}. The vertices lie on a \textit{coarse} grid while the edges lie on a \textit{fine} grid extending the coarse grid. It may appear that an orthogonal drawing may convey some structural information, so \textit{smoothening} those edges is of interest. In this thesis, we focus on the smoothening of Kandinsky drawings by introducing circular arcs, inspired by \textit{Lombardi drawings} as illustrated in Figure \ref{im:lombardi_ex}\cite{lombardi_src1}\cite{lombardi_src2}. 
%\begin{figure}[H]
%	\centering
%	\begin{subfigure}{0.45\textwidth}
	%		\centering
	%		\includegraphics[width=0.4\linewidth,page=1]{includegraphics/introduction-example}
	%		\caption{Orthogonal drawing}\label{im:introduction_ex1}
	%	\end{subfigure}
%	\begin{subfigure}{0.45\textwidth}
	%		\centering
	%		\includegraphics[width=0.4\linewidth,page=2]{includegraphics/introduction-example}
	%		\caption{Smoothened drawing}\label{im:introduction_ex2}
	%	\end{subfigure}
%	\caption{Smoothening a drawing for aesthetic appeal}
%\end{figure}
%By postprocessing an input drawing like illustrated in Figure \ref{im:introduction_ex1} and \ref{im:introduction_ex2}, we also have to consider possible shape alternations. It is desirable that the orientation of the vertices is preserved, meaning that e.g. a metro map can still be read reasonably after the smoothening process\cite{metro1}.\\
%However, it is a priori not guaranteed that there is a smoothening for every input Kandinsky drawing with a reasonable complexity increase. The introduction of circular arcs might arise some conflicts in sense of planarity. Dealing with postprocessing algorithms, we have to focus on new area bounds and the behaviour of the edge complexity in order to quantify the resulting quality of the smoothened drawing.

\section{Preliminaries}
\subsection{Definitions}
%A graph is a 2-tuple consisting of a vertex set and an edge set. The visualization, however, has to be drawn in some kind of way. Investigating the drawing of a graph needs to include several constraints, for example, \emph{how} we want the drawing and \emph{where} we want to draw on. These constraints can be described mathematically.\\
As otherwise mentioned, a \emph{graph} $G=(V,E)$ is a tuple consisting of two sets - the set of vertices and the set of edges. An \emph{edge} $e = (v,w), v,w \in V$ is a tuple and describes a connectivity relation between two vertices.
% Undirected
Unless otherwise mentioned, the graphs are \emph{undirected}, meaning that the edge $(u,v)$ is identical to the edge $(v,u)$.
% Face
A \emph{face} is a maximal open region of the plane bounded by edges. 
% degree of G
The \emph{degree} of a vertex states the amount of edges incident to the vertex.
%  The \emph{degree} of a graph $G$ is the maximum of the degree of its vertices. 
% Grid

% Embedding
An \emph{embedding} of $G$ is the collection of counter-clockwise circular orderings of edges around each vertex of $V$.
% Drawing
A \emph{drawing} $\Gamma$ of a graph $G$ is a function, where each vertex is mapped on a unique point $\Gamma(v)$ in the plane and each edge is mapped on an open Jordan curve $\Gamma(e)$ ending in its vertices.
% Planarity
A graph is \emph{planar} if and only if there exists a crossing-free representation in the plane. 
\cite[Page 100]{DBLP:books/daglib/0023376}
% Area

% Straight-line drawing

% Box drawing

% Poly line drawing

% Euclidian distance

% Edge length

% ratio

\subsection{Graph Classes}
% tree

% k-nary tree

% series parallel graph

% outerplanar graph

% 2-tree

\subsection{Tools}

% SPQR Tree

\include{include/Kandinsky}
\section{Saving measures}
\subsection{Introduction}
With our existing approaches we are able to translate any Kandinsky drawing with arbitrary degree to a smooth orthogonal layout with a complexity increase by 2 for larger polyedges in $\Rho(n^2)\times\Rho(n)$ area. The question arises whether it was possible to even save some bends or find a method to stretch with less area requirements in the worst case. In the following section, we examine possibilities to decrease the complexity of polyedges. Furthermore we try to lower the area upper bound in a possible tradeoff with some more bends.
\subsection{Edge Complexity Bounds}
Reconsider the drawing given in Figure \ref{im:kandinsky_bends2}. Due to the fact that the previous results considering 4-planar graphs are also holding for graphs of arbitrary degree, the results are applicable.\\
\begin{figure}[H]
	\centering
	\begin{subfigure}{0.6\linewidth}
		\centering
		\includegraphics[width=0.7\textwidth,page=2]{includegraphics/kandinsky_bends_arbitrary.pdf}
	\end{subfigure}
	\caption{Recall Figure \ref{im:kandinsky_bends2}}\label{im:recall1}
\end{figure}
\begin{theorem}
	Every Kandinsky drawing of a complexity-3 graph with arbitrary degree can be postprocessed to a complexity-4 smooth orthogonal layout in $\Rho(n^2)\times\Rho(n)$ area.
\end{theorem}
\begin{proof}
	Just as the proof of Theorem \ref{th:3to4} holds, the Kandinsky bends are not erasable, leading to a staircase situation of complexity 3. The boxing brings along one more bend, resulting in a polyedge of complexity 4.
\end{proof}
The difference in our new situation is that we are able to reposition the polyedges on the ports. By having a 4-planar graph interpreted in Kandinsky, we are able to connect up to four edges on a single port. This gives us a new opportunity to optimize the drawings.
\begin{remark}
	The \textit{bend or end} property complicates the port reassignment possibilities.
\end{remark}
If we recall Figure \ref{im:kandinsky_bends2}, we could try to alter the port the edge is connected to. Unfortunately, the uniformity of the vertex boxes lead to a possible collision illustrated in Figure \ref{im:bend_or_end_collision}. By doing so, we would further increase the complexity of the edge.
\begin{figure}[H]
	\centering
	\begin{subfigure}{0.3\linewidth}
		\centering
		\includegraphics[width=0.4\textwidth,page=3]{includegraphics/kandinsky_bends_arbitrary.pdf}
		\caption{\textit{bend or end} collision}\label{im:bend_or_end_collision}
	\end{subfigure}
	\begin{subfigure}{0.3\linewidth}
		\centering
		\includegraphics[width=0.4\textwidth,page=4]{includegraphics/kandinsky_bends_arbitrary.pdf}
		\caption{Complexity increase}\label{im:bend_or_end_comp_increase}
	\end{subfigure}
	\caption{Bend or end property complicates matters}\label{im:bend_or_end}
\end{figure}
But on the other hand, the circular arcs are flexible enough to achieve a possible complexity decrease.
\begin{figure}[H]
	\centering
	\begin{subfigure}{0.6\linewidth}
		\centering
		\includegraphics[width=0.7\textwidth,page=5]{includegraphics/kandinsky_bends_arbitrary.pdf}
	\end{subfigure}
	\caption{SMOG complexity decreases}
\end{figure}
However, this is not always possible, as we see in the following Figure:
\begin{figure}[H]
	\centering
	\begin{subfigure}{0.45\linewidth}
		\centering
		\includegraphics[width=0.45\textwidth,page=1]{includegraphics/port_reassignment_not_possible.pdf}
		\caption{}
	\end{subfigure}
	\begin{subfigure}{0.45\linewidth}
		\centering
		\includegraphics[width=0.7\textwidth,page=2]{includegraphics/port_reassignment_not_possible.pdf}
		\caption{}
	\end{subfigure}
	\caption{This might just be a candidate with negative result}\label{im:port_reassignment_not_possible}
\end{figure}
An approach to find candidate edges for a port reassignment could lie in \grqq half-bends\grqq~and diagonal segments, seen in the $L$ shape and the $T$ shape of the \textit{Kandinsky drawings with almost-empty faces} (Podevsaef drawings in short). In our first approach, we mainly focus on complexity-3 zig zags in the original Kandinsky Model.
\subsubsection{Podevsaef drawings}
Recalling definition \ref{def:podevsaef}, Podevsaef drawings mainly differ from SMOGs in their diagonal segments and 135\degree~bends. The approach to create a Podevsaef drawing is pretty similar to the smooth orthogonal case. At first, the planar Kandinsky drawing of the original graph $G$ is computed which then gets modified by a modification of the \textit{Topology Shape Metrics} algorithm (\cite[p. 4]{podevsaef}). The usage of diagonal segments enable us to illustrate triangular faces with two 135\degree~bends rather than 90\degree~bends. This solution for the bend or end property difficulty enables us to illustrate \textit{almost empty faces}, still inheriting bounding edges with a 0\degree~difference but not orthogonally, therefore \textit{almost} empty. The following solutions were found for $L$ shaped and $T$ shaped triangles:
\begin{figure}[H]
	\centering
	\begin{subfigure}{0.49\linewidth}
		\centering
		\includegraphics[width=0.471\textwidth,page=1]{includegraphics/L-t-shape_candidates.pdf}
		\caption{}
	\end{subfigure}
	\begin{subfigure}{0.49\linewidth}
		\centering
		\includegraphics[width=0.7\textwidth,page=2]{includegraphics/L-t-shape_candidates.pdf}
	\caption{}
\end{subfigure}
\caption{$L$ and $T$ shapes with almost empty faces}	
\end{figure}
Inspired by this illustration, we look for candidates for possible circular arc substitutions. Both examples from Figure \ref{im:recall1} and Figure \ref{im:port_reassignment_not_possible} might suit for a port reassignment. Although the complexity decrease possibility differs, they both share the achievable planar diagonal half-bend substitution after the application of the stretching technique.
\begin{figure}[H]
	\centering
	\begin{subfigure}{0.2\linewidth}
		\centering
		\includegraphics[width=0.8\textwidth,page=3]{includegraphics/L-t-shape_candidates.pdf}
		\caption{}\label{im:zig-zag_candidates1}
	\end{subfigure}
	\begin{subfigure}{0.39\linewidth}
		\centering
		\includegraphics[width=0.7\textwidth,page=4]{includegraphics/L-t-shape_candidates.pdf}
		\caption{}\label{im:zig-zag_candidates2}
	\end{subfigure}
	\begin{subfigure}{0.39\linewidth}
		\centering
		\includegraphics[width=0.7\textwidth,page=5]{includegraphics/L-t-shape_candidates.pdf}
		\caption{}\label{im:zig-zag_candidates3}
	\end{subfigure}
	\caption{Complexity-3 zig zags examined}\label{im:zig-zag_candidates}	
\end{figure}
\begin{lemma}If a polyedge representation $\Gamma_e$ is zig-zag shaped in Kandinsky (Figure \ref{im:zig-zag_candidates1}) and the port reassignment and circular arc substitution decrease the complexity in the resulting planar SMOG representation (Figure \ref{im:zig-zag_candidates2}), then there is a planar Podevsaef representation with two 135\degree~bends (Figure \ref{im:zig-zag_candidates3}).\label{lem:zig-zag_candidates}
\end{lemma}
So, according to lemma \ref{lem:zig-zag_candidates}, if we look for the possibility of planar Podevsaef polyedge representation like illustrated in figure \ref{im:zig-zag_candidates3} in a polyedge, the port reassignment of one of the vertices might just decrease the complexity. But as we seen, not in every case.
\subsubsection{Using the fragmentation}
Another approach is to use the optimal fragmentation regarding a polyedge to determine whether it was possible to save some bends. The fragmentation itself does not consider the horizontal or vertical alignment of segments in the plane. The following example will motivate the next lemma:
\begin{lemma}
	If the optimal fragmentation of a polyedge contains a fragment of length one in between two other fragments and its line segment is vertical, then the complexity does not increase at this incompatible fragment.
\end{lemma}
\begin{proof}
	Consider the alternating fragment consisting of a single vertical segment in the optimal fragmentation (Figure \ref{im:vertical_fragment2}). Then, the fragments adjacent to it are uniform. Those fragments share the same turn direction because in the uniform-only fragmentation the fragments are alternating their direction of turns. Consider that the original fragment was of length two without the recheck. The next fragment is of length at least three because the fragmentation algorithms have been shifting a second segment. This particular situation enables us to substitute the vertical segment with a half-circular arc due to the same turns of the segments before (Figure \ref{im:vertical_fragment3}). 
\begin{figure}[H]
	\centering
	\begin{subfigure}{0.33\linewidth}
		\centering
		\includegraphics[width=0.25\textwidth,page=1]{includegraphics/vertical_fragment_1.pdf}
		\caption{}\label{im:vertical_fragment1}
	\end{subfigure}	
	\begin{subfigure}{0.33\linewidth}
		\centering
		\includegraphics[width=0.25\textwidth,page=2]{includegraphics/vertical_fragment_1.pdf}
		\caption{}\label{im:vertical_fragment2}
	\end{subfigure}
	\begin{subfigure}{0.32\linewidth}
		\centering
		\includegraphics[width=0.45\textwidth,page=3]{includegraphics/vertical_fragment_1.pdf}
		\caption{}\label{im:vertical_fragment3}
	\end{subfigure}
	\caption{Illustration of the alternating fragment exception}\label{im:vertical_fragment}
\end{figure}
\end{proof}
\subsection{Area Bounds}
In this section, we will examine the possiblities of area bound optimization. Suppose that the original orthogonal drawing is very large and contains a large number of vertices, it is of interest to find a way to lower the upper bound. In the first approach, we will erase redundancies in a drawing with a plane sweep method. In the second approach, we will substitute the circular arcs with ellipses or specific segment combinations in order to lower the upper bound by $\Rho(\sqrt{n})$, which may increase the lower bound of the edge complexity on the other hand.
\subsubsection*{Plane sweep erasing}
The stretching technique does not increase the edge complexity excessively. On the other hand, it may appear that the horizontal area expansion of a drawing is unnecessarily big. In this section, a linear runtime plane sweep may be a stable solution regarding area and even edge complexity retrenchment. Consider the following example:
\begin{figure}[H]
	\centering
	\begin{subfigure}{0.33\linewidth}
		\centering
		\includegraphics[width=0.4\textwidth,page=1]{includegraphics/plane_sweep_save.pdf}
		\caption{}
	\end{subfigure}	
	\begin{subfigure}{0.33\linewidth}
		\centering
		\includegraphics[width=0.8\textwidth,page=2]{includegraphics/plane_sweep_save.pdf}
		\caption{}
	\end{subfigure}
	\begin{subfigure}{0.32\linewidth}
		\centering
		\includegraphics[width=0.4\textwidth,page=3]{includegraphics/plane_sweep_save.pdf}
		\caption{}
	\end{subfigure}

	\caption{One plane sweep could eradicate redundant area and even segments without causing any damage}\label{im:sweep_save}
\end{figure}
As you can see in Figure \ref{im:sweep_save}, an orthogonal Kandinsky drawing of a triangle gets transferred to a smooth orthogonal drawing in the Fixed Shape Model. The vertical orange-colored lines show the possible saving of area. Even one horizontal segment is eliminated in the process, reducing the complexity of the drawing to one.
\begin{definition}
	There is a plane sweep algorithm which eliminates unnecessary space and may even save some segments in linear runtime regarding the size of the drawing. The vertical segments do not have to be considered by the plane sweep.
\end{definition}
This plane sweep algorithm identifies the presence of \textit{vertices} and \textit{circular arcs} as an event. Obviously, it is undesirable to interfere with circular arcs while cutting the drawing. First we show, that considering vertices, circular arcs and horizontal line segments is sufficient.\\
The \textit{vertical line segments} are not to be considered because either they end in one or two vertices. This means that vertices are sufficient in this case to be seen. If a vertical line segment is not connected to any vertex, then, by definition of the SMOG Model, they are connected to circular arcs which are also considered by the sweep line.\\
The horizontal segments are part of the events in order to determine which segments can be cut between two events. The plane sweep iterates from the left side to the right and is able to delete unnecessary horizontal redundancy with one sweep. Therefore, this plane sweep will run in linear runtime regarding the size of the drawing.
\begin{lemma}
	This area saving plane sweep is able to save up to $\Rho(\sqrt{w})$ area, where $w$ is the width of the input drawing and it potentially lowers the overall complexity of a drawing.
\end{lemma}
\begin{proof}
	Consider following orthogonal drawing of the graph:
	\begin{figure}[H]
		\centering
		\begin{subfigure}{0.45\linewidth}
			\centering
			\includegraphics[width=0.5\textwidth,page=1]{includegraphics/plane_sweep_linear_saving.pdf}
			\caption{}\label{im:ortho_max_saving1}
		\end{subfigure}	
		\begin{subfigure}{0.4\linewidth}
			\centering
			\includegraphics[width=0.5\textwidth,page=2]{includegraphics/plane_sweep_linear_saving.pdf}
			\caption{}\label{im:ortho_max_saving2}
		\end{subfigure}
	\caption{Orthogonal drawing with maximal saving possibilites}\label{im:ortho_max_saving}
	\end{figure}
The dotted triangle of figure \ref{im:ortho_max_saving1} consists of $\Rho(n)\times\Rho(n)$ vertices as illustrated in figure \ref{im:ortho_max_saving2}. Applying the stretching technique and circular arc substitution, the resulting SMOG drawing is of size $\Rho(n^2)\times\Rho(n)$.
	\begin{figure}[H]
	\centering
	\begin{subfigure}{0.45\linewidth}
		\centering
		\includegraphics[width=0.7\textwidth,page=3]{includegraphics/plane_sweep_linear_saving.pdf}
		\caption{}\label{im:ortho_max_saving3}
	\end{subfigure}	
	\begin{subfigure}{0.4\linewidth}
		\centering
		\includegraphics[width=0.7\textwidth,page=4]{includegraphics/plane_sweep_linear_saving.pdf}
		\caption{}\label{im:ortho_max_saving4}
	\end{subfigure}
	\caption{SMOG drawing with maximal saving possibilites}
\end{figure}
In figure \ref{im:ortho_max_saving4}, the orange dotted lines indicate where the area saving plane sweep triggers an event. Notice that there are $\Rho(n)$ events.
	\begin{figure}[H]
	\centering
	\begin{subfigure}{0.45\linewidth}
		\centering
		\includegraphics[width=0.5\textwidth,page=1]{includegraphics/plane_sweep_linear_saving.pdf}
		\caption{}
	\end{subfigure}	
	\begin{subfigure}{0.4\linewidth}
		\centering
		\includegraphics[width=0.5\textwidth,page=5]{includegraphics/plane_sweep_linear_saving.pdf}
		\caption{}\label{im:ortho_max_saving5}
	\end{subfigure}
	\caption{SMOG drawing after applying the plane sweep}
\end{figure}
After cutting area redundancies, we decrease the complexity of the graph to one and save $\Rho(n)$ area horizontally.
\end{proof}
However, the plane sweep will not work properly in all scenarios, as the following example will show:
\begin{figure}[H]
	\centering
	\begin{subfigure}{0.5\linewidth}
		\centering
		\includegraphics[width=0.6\textwidth,page=1]{includegraphics/plane_sweep_not_working.pdf}
	\end{subfigure}
	\caption{The plane sweep will not reduce anything}\label{im:plane_sweep_bad_example1}
\end{figure}
Consider the drawing in Figure \ref{im:plane_sweep_bad_example1} with a unit length of 1. While iterating from left to right, there will be a vertex as a new event with unit length difference to the last event. However, this drawing can be optimized regarding horizontal area as shown in figure \ref{im:plane_sweep_bad_example2}.
\begin{figure}[H]
	\centering
	\begin{subfigure}{0.5\linewidth}
		\centering
		\includegraphics[width=0.6\textwidth,page=2]{includegraphics/plane_sweep_not_working.pdf}
	\end{subfigure}
	\caption{Optimized drawing derived from figure \ref{im:plane_sweep_bad_example1}}
	\label{im:plane_sweep_bad_example2}
\end{figure}
In order to find a solution for this situation, a new approach based on the \textit{4M} algorithm is formulated.
\subsubsection{Modified 4M - Moving}
Recalling section \ref{section:4M}, the 4M algorithm consists of multiple operations. In this section, we will consider the \textit{moving} operation. 
\subsubsection*{Model consistency}
It is to examine whether the \textit{4M} operations were suitable for drawings which underly the Kandinsky model. In the Kandinsky model, the vertices are represented with boxes of uniform size. In the orthogonal 4M model, the boxes have to overlap the grid lines by $\frac{\lambda}{4}$, and the rectangles were of size $\left(w\lambda-\frac{\lambda}{2} \right)\times \left( h\lambda-\frac{\lambda}{2}\right);w,h\in\mathbb{N}$. By setting $w = h = 1$, we achieve a consistency with the Kandinsky model and therefore can continue examining a 4M algorithm modification.
\subsubsection*{The modification}
The focus lies in the horizontal area saving. The moving line $J$ will still be directed and starts above to topmost object and ends below the bottommost object of the drawing. Considering orthogonal graphs, $J$ only crosses horizontal lines from top to bottom. In the smoothened Kandinsky case, quarter circular arcs traverse in height and width - so they also have to be crossed for total area savings. 
\begin{figure}[H]
	\centering
	\begin{subfigure}{0.4\linewidth}
		\centering
		\includegraphics[width=0.4\textwidth,page=1]{includegraphics/4M_Moving_arcs.pdf}
		\caption{$J$ crossing the drawing}
	\end{subfigure}
	\begin{subfigure}{0.4\linewidth}
		\centering
		\includegraphics[width=0.2\textwidth,page=2]{includegraphics/4M_Moving_arcs.pdf}
		\caption{Circular arc substitution}
	\end{subfigure}
	\caption{A first, easy example}\label{im:arc-crossing_example1}
\end{figure}
As you can see in figure \ref{im:arc-crossing_example1}, a quarter circular arc with radius $r$ can be reduced in width by substituting with a circular arc with radius $r-c\cdot\lambda$ and a vertical segment of length $c\cdot\lambda$. $\lambda$ describes the unit length, $c$ describes the amount of unit reducings possible along $J$. By this method, the complexity may increase by one per crossed quarter circular arc. 
\begin{figure}[H]
	\centering
	\begin{subfigure}{0.4\linewidth}
		\centering
		\includegraphics[width=0.4\textwidth,page=3]{includegraphics/4M_Moving_arcs.pdf}
		\caption{$J'$ crossing the drawing}\label{im:arc-crossing_example2a}
	\end{subfigure}
	\begin{subfigure}{0.4\linewidth}
		\centering
		\includegraphics[width=0.265\textwidth,page=4]{includegraphics/4M_Moving_arcs.pdf}
		\caption{Quarter circular arc substitution}
	\end{subfigure}
	\caption{A second example}\label{im:arc-crossing_example2}
\end{figure}
In this example, $J'$ crosses two consecutive quarter circular arcs in a drawing of size $6\times6$. The result is a drawing of size $4\times6$, with an edge complexity of three. A verticel line segment of length one is introduced per substituted arc, merged to a vertical line segment of length two.\\
To further decrease the area of the drawing, a horizontal path from the leftmost object to the rightmost one would reduce and even eliminate the vertical line segments crossed by $J'$.
\begin{figure}[H]
	\centering
	\begin{subfigure}{0.4\linewidth}
		\centering
		\includegraphics[width=0.7\textwidth,page=5]{includegraphics/4M_Moving_arcs.pdf}
		\caption{$J'$ crossing the drawing}
	\end{subfigure}
	\begin{subfigure}{0.4\linewidth}
		\centering
		\includegraphics[width=0.7\textwidth,page=6]{includegraphics/4M_Moving_arcs.pdf}
		\caption{Circular arc substitution}
	\end{subfigure}
	\caption{Vertical area saving}\label{im:arc-crossing_example3}
\end{figure}
The area bounds of the drawing from Figure \ref{im:arc-crossing_example2a} can further be reduced by a horizontal moving line $J'$, resulting in a $4\times4$ drawing. In order to do so, one has to \textit{reduce the line segment between quarter circular arcs first}. This results in a possible complexity decrease, in this case a drawing of complexity two. \\
Recall the elongation of horizontal edges crossed by the original moving line $J$ with an upward direction. What, if a quarter circular arc is crossed with an upward piece of $J'$? Then, the idea is to add a horizontal line to the quarter circular arc correspondingly with unit length. Consider the following example:
\begin{figure}[H]
	\centering
		\begin{subfigure}{0.4\linewidth}
		\centering
		\includegraphics[width=0.7\textwidth,page=1]{includegraphics/4M_Moving_arcs_upward.pdf}
		\caption{$J'$ crossing the drawing}
	\end{subfigure}
	\begin{subfigure}{0.4\linewidth}
		\centering
		\includegraphics[width=0.6\textwidth,page=2]{includegraphics/4M_Moving_arcs_upward.pdf}
		\caption{Additional horizontal segments}
	\end{subfigure}
	\caption{The additional horizontal line segment increases the complexity of the drawing to two}
\end{figure}
This idea implies the modified moving approach.
\begin{definition}[modified moving line $J'$]
	An area-saving moving line $J'$ for a smoothened drawing $\Gamma ''$ is a line that fulfills the following conditions:
	\begin{itemize}
		\item $J'$ is directed and consists of horizontal and vertical segments
		\item $J'$ starts above the topmost object of $\Gamma ''$ and ends below the bottommost object of $\Gamma ''$
		\item $J'$ does not intersect any vertical edge of $\Gamma ''$
		\item In a polyedge $e$, either quarter circular arc segments or horizontal line segments might be crossed by $J'$ 
		\item Every horizontal edge of $\Gamma''$ that is intersected by a piece of $J'$ which is directed downward has a finite length larger than or equal to two. If this horizontal line segment is connected to a circular arc segment, then it has a finite length larger or equal to one.
		\item Every circular arc segment of $\Gamma''$ that is intersected by a piece of $J'$ which is directed downward has a finite radius larger than or equal to two.
	\end{itemize}
\end{definition}
Since $\Gamma''$ is planar, its dual graph is well-defined and planar. $J'$ can easily be found with \textit{depth first search} in the planar dual graph. 
Then, after finding $J'$, a smaller drawing can be computed the following way:
\begin{itemize}
	\item The length of every edge of $\Gamma''$ that is crossed by $J'$ in downward direction is decreased by one unit
	\item Simultaneously, the length of every edge crossed by $J'$ in upward direction is increased by one unit
	\item If a quarter circular arc is crossed by $J'$ in downward direction, then the arc is substituted by a quarter circular arc with a smaller radius by one unit with a vertical segment of unit length (refer to Figure \ref{im:arc-crossing_example1})
	\item If a quarter circular arc is crossed by $J'$ in upward direction, then a horizontal line segment of unit length is appended to the quarter circular arc correspondingly
\end{itemize}
Note, that this modified moving algorithm separates the drawing in two parts according to $J'$ and then moving one part closer to the other one. The direction of area saving can be altered by switching the roles of vertical and horizontal line segments and finding a horizontal directed path $J'$ instead of a vertical one.
\subsubsection*{Correctness}
Consider a quarter circular arc $c$ with radius $r$ crossed by $J'$ in a smoothened drawing. By substituting $c$ with a corresponding horizontal and vertical line segment, $h_c$ and $v_c$ with length $r$, the original moving algorithm preserves the planarity of the drawing.
\\
If the original moving line $J$ crosses $h_c$ downwards, then $h_c$ will be shorter in length than $v_c$. The substitution with a smaller circular arc with radius $|h_c|$ will increase the total edge complexity since $|v_c| > |h_c|$.
\\If the original moving line $J$ crosses $h_c$ upwards, then $h_c$ will be larger in length than $v_c$. The substitution with the input circular arc with radius $|v_c|$ increases the complexity since $|v_c| < |h_c|$. 
\\The area necessary for circular arc substitution is already proven in the previous section, preserving the planarity.
\\All we need to prove is the choice of segment to cross, when a horizontal line segment $h$ is appended to a circular arc $c$. We will be able to reduce $h$ successively until $h$ may be completely erased because $c$ is still unaltered and travels the width and height necessary for the drawing. Thus, we will choose $h$ for being crossed by $J'$, possibly achieving an edge complexity reduction.
\subsubsection{Circular arc substitution}
The circular arcs used in smooth orthogonal drawings have a height and width of $r$ and are the main reason for the quadratic total width in the worst case. In this section, we examine the possibilites of saving some space by substituting the circular arcs used with different segments. In our first approach, we will use ellipses to guarantee a width of $\sqrt{r}$, saving at least $\sqrt{n}$ area requirements. However, the aesthetics may suffer for large values. In our second approach, we will substitute the circular arc with a combination of a smaller circular arc and a vertical segment, also demanding $\sqrt{r}$ width. The aesthetics may be preserved but this definitely will increase the edge complexity of any drawing.
\subsubsection*{Ellipses}
Using a quarter of an ellipse, we could achieve a guaranteed width of $\sqrt{r}$, resulting in drawings of size $\Rho(n\cdot\sqrt{n})\times\Rho(n)$. We will take a look at following example equation given for an ellipse:
\begin{align}
\frac{x^2}{5} + \frac{y^2}{25} = 1&&x,-y \in \mathbb{R}_+\label{eq:ellipse_example}
\end{align}
\begin{figure}[H]
	\centering
	\begin{subfigure}{0.4\textwidth}
		\centering
		\includegraphics[width=0.7\linewidth,page=1]{includegraphics/ellipse_example_values.pdf}
	\end{subfigure}
	\caption{Illustration of equation \ref{eq:ellipse_example}}\label{im:ellipse5}
\end{figure}
For the orientation of the ellipse arcs, we will pick the values of $x$ and $y$ adequately. The extreme values of equation \ref{eq:ellipse_example} are $(0,-5)$ and $(\sqrt{5},0)$, it would fit right in instead of a circular quarter arc with radius $5$. If $5$ was the longest vertical segment, it would be sufficient to stretch the drawing by $\sqrt{5}$.
This implies - given a height $l$, serving as radius for the original circular arcs - following equation:
\begin{align}
\frac{x^2}{l}+\frac{y^2}{l^2} = 1&&x,-y \in \mathbb{R}_+\label{eq:ellipse_general}
\end{align}
With following extreme values: $(0,-l)$ and $(\sqrt{l},0)$. For every vertical segment of length $l'$, we could compute the ellipse given by equation \ref{eq:ellipse_general} and by gauging our values for $x$ and $y$ we could pick the right orientation of the arc from its appropriate quadrant. 
\subsubsection*{Correctness}
The argument is very similar to the \grqq Boxing\grqq. Instead of a square box, we will deal with rectangles of size $w\times h$. Let $l$ be the longest vertical segment in a given orthogonal drawing $\Gamma_G$. By stretching the drawing by the factor of $\sqrt{|l|}$, every vertical line segment $v$ now has a free rectangle area of size $\sqrt{|l|} \times |v|$ left and right from it. Therefore, $v$ has a free rectangle area of size $\sqrt{|v|} \times |v|$, since $|v| \leq |l| \Rightarrow |\sqrt{v}| \leq |\sqrt{l}|$, guaranteeing the free area for the ellipse arc substitution and a drawing of size $\Rho(n\cdot\sqrt{n})\times\Rho(n)$ in the worst case.
%Utilizing the strict monotonicity of the square root function, we would now be able to stretch the original Kandinsky drawing by the square root of the longest vertical segment $\sqrt{l}$, guaranteeing $\Rho(n\cdot\sqrt{n})\times\Rho(n)$ area.
\subsubsection*{Readability}
Using arcs from ellipses seems to be a good idea at first since we can actually save space, even in the worst case. But using those arcs decrease the readability of a drawing, the bigger the longest vertical segment gets.
\begin{figure}[H]
	\centering
	\begin{subfigure}{0.3\textwidth}
		\centering
		\includegraphics[width=0.7\linewidth,page=1]{includegraphics/ellipse_example_values.pdf}
		\caption{$|l'| = 5$}
	\end{subfigure}
	\begin{subfigure}{0.3\textwidth}
		\centering
		\includegraphics[width=0.7\linewidth,page=2]{includegraphics/ellipse_example_values.pdf}
		\caption{$|l'| = 100$}
	\end{subfigure}
	\begin{subfigure}{0.3\textwidth}
		\centering
		\includegraphics[width=0.7\linewidth,page=3]{includegraphics/ellipse_example_values.pdf}
		\caption{$|l'| = 1000$}
	\end{subfigure}
	\caption{Illustration of increasing values for the vertical segment result in very steep ellipse arcs}
\end{figure}
\subsubsection{Combination of circular arcs and vertical segments}
Maybe a slightly different approach would be to set the width traveled by $\sqrt{l'}$, whereas $l'$ describes the length of the horizontal segment. A combination of a circular arc with radius $\sqrt{l'}$ and a vertical segment of length $l'-\sqrt{l'}$ would illustrate the outgoing edge more precisely. 
\begin{figure}[H]
	\centering
	\begin{subfigure}{0.3\textwidth}
		\centering
		\includegraphics[width=0.7\linewidth,page=1]{includegraphics/vSMOG_example_values.pdf}
		\caption{$|l'| = 5$}
	\end{subfigure}
	\begin{subfigure}{0.3\textwidth}
		\centering
		\includegraphics[width=0.7\linewidth,page=2]{includegraphics/vSMOG_example_values.pdf}
		\caption{$|l'| = 100$}
	\end{subfigure}
	\begin{subfigure}{0.3\textwidth}
		\centering
		\includegraphics[width=0.7\linewidth,page=3]{includegraphics/vSMOG_example_values.pdf}
		\caption{$|l'| = 1000$}
	\end{subfigure}
	\caption{Illustration of increasing values for the vertical segment with the combination of a circular arc and a vertical segment}
\end{figure}
On one hand, we would still take $\Rho(n\cdot\sqrt{n})\times\Rho(n)$ area in the worst case but on the other hand the complexity in this altered smooth orthogonal layout would significantly increase.
\section{Extensional Work}
% TODO What is this section about?

\section{An Example}
Consider the following input graph drawing $\Gamma_G$:
\begin{figure}[H]
	\centering
	\begin{subfigure}{\textwidth}
		\centering
		\includegraphics[width=0.2\linewidth,page=1]{includegraphics/big-example}
	\end{subfigure}
	\caption{Example input drawing $\Gamma_G$}
\end{figure}
For measurement, the unit of length lies in the width of the illustrated boxes. The longest vertical segment lies on the left of the drawing and is of length 10. The complexity of the drawing equals three. The drawing is of size $7\times11$. 
\begin{figure}[H]
	\centering
	\begin{subfigure}{\textwidth}
		\centering
		\includegraphics[width=0.8\linewidth,page=2]{includegraphics/big-example}
	\end{subfigure}
	\caption{$\Gamma_G$ after the stretching technique application}
\end{figure}
\begin{figure}[H]
	\centering
	\begin{subfigure}{\textwidth}
		\centering
		\includegraphics[width=0.8\linewidth,page=3]{includegraphics/big-example}
	\end{subfigure}
	\caption{$\Gamma_G$ after the circular arc substitution}
\end{figure}
After stretching and smoothening the complexity of the resulting SMOG drawing rises from 3 to $\left\lfloor\frac{3}{2}\cdot 3\right\rfloor = 4$. The size of the drawing equals $37\times11$. The drawing inherits some redundancy which we can get rid of thanks to the saving measures.
\begin{figure}[H]
	\centering
	\begin{subfigure}{\textwidth}
		\centering
		\includegraphics[width=0.6\linewidth,page=4]{includegraphics/big-example}
	\end{subfigure}
	\caption{$\Gamma_G$ after the saving plane sweep application}
\end{figure}
The drawing lost 51,4\% in horizontal area requirements resulting in a drawing with $18\times11$ area bounds. Also, the complexity decreased from 4 to 3.
\begin{figure}[H]
	\centering
	\begin{subfigure}{\textwidth}
		\centering
		\includegraphics[width=0.6\linewidth,page=7]{includegraphics/big-example}
	\end{subfigure}
	\caption{$\Gamma_G$ after the port reassignment}
\end{figure}
The port reassignment results in a smoothened drawing of complexity 2 and the area consumption can further be reduced, resulting in a drawing of size $16\times 11$.
\section{Future Work}\label{section:future_work}

\begin{itemize}
	\item compress the area for any $k$-ary tree with optimal ratio
	\item Is log square a sharp upper bound
	\item implementation of maximal SP graph drawing
	\item find a new approach for 3-trees
\end{itemize}
\section{Acknowledgements}
I would like to thank Prof. Michael Kaufmann and Henry Förster for instructing this final thesis. Further I appreciate helpful discussions with Dr. Michalis Bekos. I want to thank especially my parents for their patience and encouragement despite hardest family issues.
%\begin{thebibliography}{9}
	\bibitem{SMOG}
	Smooth Orthogonal Layouts,
	\textit{Journal of Graph Algorithms and Application vol. 17, p.~575-595}\\
	Bekos, Kaufmann, Kobourov, Symvonis,
	2013.
	\bibitem{SMOcti_rel}
	On Smooth Orthogonal and Octilinear Drawings:
	\textit{Relation, Complexity and Kandinsky Drawings}\\
	Bekos, Förster, Kaufmann,
	submitted September 2017.
	\bibitem{SMOcti}
	On Smooth Orthogonal and Octilinear Drawings, \textit{Preprint Edition}\\
	Bekos, Förster, Kaufmann.
	\bibitem{Ortho}
	Planar Orthogonal and Polyline Drawing Algorithms,
	\textit{Handbook of Graph Drawing and Visualization,, page 223-245},
	Duncan, Goodrich,
	2013.
	\bibitem{slideplayer}
	Constraints In Orthogonal Graph Drawing, Stroh, Retrieved from \url{http://slideplayer.org/slide/660000/}, March 2018.
	\bibitem{podevsaef}
	The Effect Of Almost-Empty Faces On Planar Kandinsky Drawings, Bekos, Kaufmann, Krug, Siebenhaller\\
	online June 2015.
\end{thebibliography}
\printbibliography
\end{document}